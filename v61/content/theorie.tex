\section{Zielsetzung}
\label{sec:Ziel}

Ziel dieses Versuches ist es den Aufbau und die Funktionsweise eines Helium-Neon Lasers zu beschrieben und näher zu untersuchen.
Konkret werden seine Stabilität, zwei TEM-Moden, seine Polarisation und seine Wellenlänge gemessen.

\section{Theorie}
\label{sec:Theorie}


\subsection{Grundlagen des He-Ne Lasers}

Jeder Laser besteht aus drei Kernkomponenten. Diese sind ein aktives Medium, eine Pumpquelle und ein optischer Resonator. Beim He-Ne Laser besteht das aktive Medium
aus einem Helium-Neon Gemisch mit einem Verhältnis von 5 zu 1. Als Pumpquelle wird an das Gemisch eine Spannung angelegt. Durch Wechselwirkungen mit den Elektronen werden
die Atome im aktiven Medium angeregt. Diese angeregten Zustände sind im Helium wesentlich langlebiger als im Neon. Die Heliumatome sind in der Lage ihren angeregten Zustand
durch einen Stoß an die Neonatome zu übertragen. Dies führt dazu, dass man den gewünschten Zustand erreicht, in dem sich mehr Elektronen im Neon in einem höheren Energieniveau
befinden als im Niedrigeren. Dieser Zustand wird Besetzungsinversion genannt.
\newline
Für die Elektronen im höheren Niveau gibt es nun zwei Möglichkeiten in ein niedrigeres Niveau zurück zu kommen. Diese werden spontane Emission und stimulierte Emission gennant.
Wie der Name vermuten lässt, können Elektronen spontan, also lediglich durch Quantenflktuation, in ein niedrigeres Energieniveau zurück fallen.
Dabei emitieren sie ein Photon mit einer Energie, die der Differenz der beiden Niveaus entspricht. Die Richtung in die das Photon emittiert wird ist hierbei beliebig.
\newline
Auch bei der stimulierten Emission wechselt das Elektron vom höheren ins niedrigere Energieniveau. Der Unterschied ist, dass dieser Vorgang durch ein einfallendes Photon ausgelößt
wird. Voraussetzung hierfür ist, dass die Energie des Photons der Differenz der Energieniveaus entspricht. Bei der stimulierten Emission wird ebenfalls ein Photon der selben Energie
emittiert, dieses bewegt sich jedoch immer parallel zum einfallenden Photon. Aus diesem Grunde ist es für die gewünschte Verstärkung des Lasers notwendig, dass die stimulierten
Emissionen der dominante Prozess sind.
\newline
\newline
Als optischer Resonator werden beim He-Ne Laser zwei Spiegel verwendet. Einer der Spiegel ist teildurchlässig, damit der Laserstrahl zu einer Seite hin ausgekoppelt werden kann.
Hiefür können die Spiegel sowohl planparallelen, als auch sphärischen sein. Auch eine Kombination ist möglich. Die Verlusste der Spiegel müssen hierbei jedoch möglichst gering sein,
um Oszillatorverhalten zu erreichen. Ein selbsterhaltender Resonator ist nur dann möglich, wenn diese Verluste kleiner sind, als die Verstärkung durch die induzierten Emissionen.
Es folgt somit für einen stabilen Resonator die Relation
\begin{equation}
    0 \leq g_1 \cdot g_2 \leq 1.
\end{equation}
\noindent
Hierbei sind die Resonatorparameter $g_1$ und $g_2$ gegeben durch
\begin{equation}
    g_\text{i} = 1 - \frac{L}{r_\text{i}}.
\end{equation}
\noindent
$L$ beschreibt hierbei die Resonatorlänge und $r_\text{i}$ die Krümmungsradien der Spiegel. Da die Resonatorlänge sehr viel größer ist, als die Wellenlänge des Lasers,
gibt es sehr viele Frequenzen, die die Resonatorbedingung einer stehenden Welle erfüllen.



\subsection{TEM-Moden}

Die möglichen Wellenlängen im Resonator werden als longitudinale Mode bezeichnet. Da die Resonatorspiegel nicht frei von Unebenheiten sind, können auch transversale
Moden auftreten. Die verschiedenen Moden eines Resonators werden als $\symup{TEM}_\text{lqp}$ klassifiziert, wobei die Indizes l und p die Knoten in x- und y-Richtung
und q die longitudinale Mode beschreiben. Die $\symup{TEM}_\text{00}$ Grundmode ist hierbei die Mode mit der höchsten Symmetrie und daher mit den geringsten Verlusten.
Ihre Intensitätsverteilung kann durch eine Gaußverteilung der Form
\begin{equation}
I(r)=I_0e^{\frac{-2r^2}{\omega^2}}
\end{equation}
beschrieben werden, wobei $I_0$ die Maximalintensität, $r$ der Abstand zur optischen Achse und $\omega$ der Strahlradius ist.
Dieser kann über
\begin{equation}
  \omega(z)=\omega_0\sqrt{1+\Bigl(\frac{\theta z}{\omega_0}\Bigr)^2},
\end{equation}
mit dem Abstand $z$ von der minimalen Strahltaillie $\omega_0$, berechnet werden.
$\theta=\frac{\lambda}{\pi}\omega_0$ beschreibt hierbei die Divergenz des Gaußschen Strahls.


\subsection{Wellenlänge}

Trifft kohärentes Licht durch ein optisches Gitter auf einen Schirm, so entsteht ein Beugungsmuster. Durch Messen des Abstandes vom Gitter zum Schirm und von den
jeweiligen Maxima zum Hauptmaximum kann mit der Gleichung
\begin{align}
    \lambda = \frac{b}{n}\cdot\sin\left(\tan^{-1}\left(\frac{d_n}{L}\right)\right).
\end{align}
\noindent
die Wellenlänge bestimmt werden.