\section{Auswertung}
\label{sec:Auswertung}
Im folgenden werden die in \autoref{sec:Durchführung} aufgenommenen Messwerte ausgewertet. Dazu sollten ursprünglich zwei Spiegelkonfigurationen getrennt
voneinander betrachtet werden, im Versuch war es jedoch nicht möglich einen Laser mit einer plan-konkaven Spiegelkonfiguration aufzubauen. Dies wird in \autoref{sec:Diskussion} näher betrachtet werden.
Die Berechnungen werden mit den Python-Erweiterungen numpy \cite{numpy}, scipy \cite{scipy} und uncertainties \cite{uncertainties} ausgerechnet und mit matplotlib \cite{matplotlib} grafisch dargestellt.

\subsection{Wellenlängenbestimmung}
\label{subsec:wellen}

Die zur Bestimmung der Wellenlänge des HeNe-Lasers aufgenommenen Messwerte sind in \autoref{tab:wellen} abgebildet, wobei für die unterschiedlichen Gitter die Abstände zum Hauptmaximum aufgetragen wurden.
Das erste Gitter wird mit $d_1$ gekennzeichnet und hat $(80\, \text{Linien}\,/\,\si{\milli\meter})$, das zweite hat $(100\, \text{Linien}\,/\,\si{\milli\meter})$, das dritte $(600\, \text{Linien}\,/\,\si{\milli\meter})$ und das
vierte Gitter besitzt $(1200\, \text{Linien}\,/\,\si{\milli\meter})$.
Die Abstände der ersten beiden Gitter zum Schirm sind jeweils die gleichen und betragen am linken Rand $L_\text{links}=\SI{67.5}{\centi\meter}$ und am rechten Rand $L_{\text{rechts}}=\SI{59.3}{\centi\meter}$, woraus sich ein
durchschnittlicher Schrimabstand von $L=\SI{63.4}{\centi\meter}$ ergibt.
Die letzten beiden Gitter haben jeweils erneut die gleichen Abstände und betragen am linken Rand $L_\text{links}=\SI{58}{\centi\meter}$ und am rechten Rand $L_{\text{rechts}}=\SI{37.5}{\centi\meter}$, woraus sich der
durchschnittlicher Schrimabstand von $L=\SI{47.75}{\centi\meter}$ ergibt.


\begin{table}[H]
  \centering
  \caption{Abstände der Interferenzmaxima der jeweiligen Gitter zum entsprechenden Maximum.}
  \label{tab:wellen}
  \begin{tabular}{c c c c c}
      \toprule
      Maximum & $d_1/\,\si{\centi\meter}$ & $d_2/\,\si{\centi\meter}$ & $d_3/\,\si{\centi\meter}$ & $d_4/\,\si{\centi\meter}$ \\
      \midrule
      1 & 2,4 & 3,2 & 8,5 & 25,2\\  
      2 & 5,0 & 6,4 & 25,3 &    \\  
      3 & 7,7 & 9,9 &       &   \\  
      4 & 10,5 & 13,2 &     &   \\
      5 & 13,1 & 16,9 &     &   \\
      6 & 16,0 & 20,8 &     &   \\
      7 & 19,0 & 24,9 &     &   \\
      8 & 22,0 & 29,7 &     &   \\
      \bottomrule
  \end{tabular}
\end{table}

\noindent
Die Wellenlängen ergeben sich über die folgende Relation,
\begin{align}
  \lambda = \frac{b}{n}\cdot\sin\left(\tan^{-1}\left(\frac{d_n}{L}\right)\right).
\end{align}

\noindent
Hierbei ist $n$ die Ordnung des Maximums, $b$ die Gitterbreite, $d$ der Abstand vom $n$-ten Maximum zum Hauptmaximum und $L$ der Abstand zwischen dem Gitter und dem Schirm.
Die Wellenlängen, die sich aus den Werten aus \autoref{tab:wellen} ergeben, sind in \autoref{tab:wellenlaengen} eingetragen.

\begin{table}[H]
  \centering
  \caption{Wellenlängen der Interferenzmaxima der jeweiligen Gitter zum entsprechenden Maximum.}
  \label{tab:wellenlaengen}
  \begin{tabular}{c c c c c}
      \toprule
      Maximum & $\lambda_1 \,/\,\si{nm}$ & $\lambda_2 \,/\,\si{nm}$ & $\lambda_3 \,/\,\si{nm}$ & $\lambda_4 \,/\,\si{nm}$\\
      \midrule
      1 & 491.376 & 502.180   & 0.221   & 0.2199  \\  
      2 & 502.355 & 514.273   & 0.148  &       \\  
      3 & 510.592 & 509.577   &       &       \\  
      4 & 505.876 & 515.136 &       &       \\
      5 & 509.779 & 519.547 &       &       \\
      6 & 512.626 & 522.231 &       &       \\
      7 & 512.230 & 530.268 &       &       \\
      8 & 472.847 & 504.090 &       &       \\
      \bottomrule
  \end{tabular}
\end{table}

\noindent
Die Mittelwerte ergeben sich zu,
\begin{align*}
  \overline{\lambda_1} &= \SI{502.210}{\nano\meter}, \\
  \overline{\lambda_2} &= \SI{514.663}{\nano\meter}, \\
  \overline{\lambda_3} &= \SI{0.185}{\nano\meter}.
\end{align*}

\noindent
Da die berechneten Wellenlängen der letzteren beiden Gitter nicht mehr im sichtbaren Spektrum liegen, werden sie bei der Berechnung eines gesamten Mittelwertes ausgelassen. Die Wellenlänge aus allen betrachteten Werten
ergibt sich demnach zu $\lambda=\SI{508.436}{\nano\meter}$.

\subsection{TEM-Moden}
\label{subsec:TEM}
Im folgenden werden die beiden im Versuch untersuchten Moden hinsichtlich der Theorie ausgewertet.

\subsubsection{$\text{TEM}_{00}$-Mode}
\label{subsubsec:grundmode}
Die Messwerte zur Auswertung der Grundmode, die in \autoref{tab:grundModeWerte} aufgetragen sind, werden in \autoref{fig:plot1} grafisch gegen eine Ausgleichskurve ausgewertet, die sich über folgende Form ergibt
\begin{align}
  I_{(0,0)}(L)=I_0\exp\left({-2\left(\frac{L-d_0}{\omega}\right)^2}\right).
\end{align}
Dabei ist $I_0$ die maximale Intensität, $d_0$ die Verschiebung der Photodiode senkrecht zur Strahlebene und $\omega$ der Strahlradius.
Die aus dem Fit berechneten Parameter lauten
\begin{align*}
  I &= \SI{63.5 \pm 1.1}{\micro\watt}, \\
  d &= \SI{4.918 \pm 0.004}{\milli\meter}, \\
  \omega &= \SI{4.918 \pm 0.004}{\micro\watt}.
\end{align*}

\begin{table}[H]
  \centering
  \caption{Messwerte zur Auswertung der TEM-Grundmode in Abhängigkeit der Position.}
  \label{tab:grundModeWerte}
  \begin{tabular}{c c}
      \toprule
      $L$ / mm & $I$ / $\si{\micro\watt}$\\
      \midrule
      -6 & 0.38 \\
      -5 & 0.75 \\
      -4 & 1.63 \\
      -3 & 3.46 \\
      -2 & 6.9 \\
      -1 & 13.7 \\
      0 & 21.5 \\
      1 & 32.1 \\
      2 & 40.2 \\
      3 & 49.5 \\
      4 & 56.7 \\
      5 & 64.6 \\
      6 & 64.1 \\
      7 & 53.4 \\
      8 & 41.1 \\
      9 & 24.8 \\
      10 & 15.5 \\
      11 & 8.77 \\
      12 & 3.62 \\
      13 & 1.24 \\
      14 & 0.5 \\
      15 & 0.23 \\
      \bottomrule
  \end{tabular}
\end{table}

\begin{figure}[H]
  \centering
  \includegraphics[width=0.75\textwidth]{plot.pdf}
  \caption{Grafische Auswertung und Fit der Messwerte zur TEM-Grundmode.}
  \label{fig:plot1}
\end{figure}


\subsubsection{$\text{TEM}_{01}$-Mode}
\label{subsubsec:Mode1}
Die Auswertung der Messwerte aus \autoref{tab:ersteModeWerte} wird analog zur Grundmode ausgewertet, wobei jedoch die Theoriekurve durch eine Funktion der Form
\begin{align}
  I_{\text{TEM}_{01}} = I_1 \frac{8\left( x - x_0\right)^2}{\omega^2} \exp \left( \frac{-\left(x - x_0\right)^2}{2\omega^2} \right)
\end{align}
beschrieben wird. Die Parameter aus dem Fit aus \autoref{fig:plot1} lauten damit
\begin{align*}
  I_1&= \SI{3.6857 \pm 0.0021}{\micro\watt}, \\
  x_0 &= \SI{0.0519 \pm 0.0014}{\milli\meter}, \\
  x_1 &= \SI{0.0264 \pm 0.0015}{\milli\meter}, \\
  \omega &= \SI{0.0264 \pm 0.0015}{\milli\meter}.
\end{align*}

\begin{figure}[H]
  \centering
  \includegraphics[width=0.75\textwidth]{plot1.pdf}
  \caption{Grafische Auswertung der Messwerte zur $\text{TEM}_{01}$-Mode.}
  \label{fig:plot1}
\end{figure}

\begin{table}[H]
  \centering
  \caption{Messwerte zur Auswertung der $\text{TEM}_{01}$-Mode.}
  \label{tab:ersteModeWerte}
  \begin{tabular}{c c}
      \toprule
      $L$ / mm & $I$ / $\si{\micro\watt}$\\
      \midrule
      -10 & 0.26 \\
      -9 & 0.45 \\
      -8 & 0.60 \\
      -7 & 1.1 \\
      -6 & 1.70 \\
      -5 & 3.14 \\
      -4 & 5.73 \\
      -3 & 9.60 \\
      -2 & 14.2 \\
      -1 & 19.1 \\
      0 & 21.4 \\
      1 & 19.8 \\
      2 & 16.2 \\
      3 & 10.4 \\
      4 & 3.9 \\
      5 & 0.86 \\
      6 & 1.93 \\
      7 & 7.52 \\
      8 & 15.68 \\
      9 & 20.9 \\
      10 & 22.6 \\
      11 & 18.6 \\
      12 & 13.8 \\
      13 & 8.4 \\
      14 & 4.5 \\
      15 & 2.1 \\
      \bottomrule
  \end{tabular}
\end{table}




\subsection{Polarisationsmessung}
\label{subsec:Polarisationsmessung}

Die Auswertung der Polarisationsmessung erfolgt analog zu denen aus den Abschnitten zuvor, indem die Messwerte gegen eine Theoriekurve gefittet werden.
Die Theoriekurve ergibt sich durch eine Funktion der Form
\begin{align}
  I(\varphi) = I_0 \sin^2(\varphi-\varphi_0).
\end{align}
Die aus \autoref{fig:plot2} ausgerechneten Parameter ergeben
\begin{align*}
  I_0 &= \SI{4.10916 \pm 0.00004}{\micro\watt}, \\
  \varphi_0 &= \SI{-2.59737 \pm 0.00010}{\degree}.
\end{align*}

\begin{figure}[H]
  \centering
  \includegraphics[width=0.75\textwidth]{plot2.pdf}
  \caption{Grafische Auswertung der Messwerte zur $\text{TEM}_{01}$-Mode.}
  \label{fig:plot2}
\end{figure}


\subsection{Stabilitätsbedingung}
\label{subsec:Stabilitätsbedingung}
Um die Stabilitätsbedingung auszuwerten, wird die maximale Resonatorlänge bestimmt, die mit der Spiegelkonfiguration erreicht werden konnte. Die theoretisch zu erreichende, maximale
Resonatorlänge $L$ ergibt sich über 
\begin{align*}
  L_{\text{max}}=r_1+r_2,
\end{align*}
wobei $r_1$ und $r_2$ der jeweilige Krümmungsradius der konkaven Spiegel sind. In der im Versuch verwendeten Spiegelkonfiguration betragen die Krümmungsradien jeweils $r=\SI{1400}{\milli\meter}$,
wodurch sich die maximale Resonatorlänge zu $L_\text{max} = \SI{280}{\centi\meter}$ ergibt.
Die Laserintensitäten in Abhängigkeit der Resonatorlängen sind in \autoref{tab:StabiWerte} eingetragen.


\begin{table}[H]
  \centering
  \caption{Messwerte zur Auswertung der Stabilitätsbedingung in Abhängigkeit der Resonatorlängen.}
  \label{tab:StabiWerte}
  \begin{tabular}{c c}
      \toprule
      $L$ / cm & $I$ / $\si{\micro\watt}$\\
      \midrule
      50 & 6.13 \\
      75 & 5.15 \\
      100 & 5.10 \\
      125 & 4.12 \\
      150 & 1.31 \\
      \bottomrule
  \end{tabular}
\end{table}

\noindent
Die maximale Länge konnte dabei nicht erreicht werden, weshalb die Gültigkeit nicht nachgewiesen werden konnte.

\subsection{Frequenzbreite}
\label{subsec:Frequenzbreite}

Die gemessenen Frequenzpeaks $f$ sind in \autoref{tab:FrequenzWerte} in Abhängigkeit der Resonatorlänge $L$ festgehalten.

\begin{table}[H]
  \centering
  \caption{Messwerte zur Auswertung der Frequenzpeaks.}
  \label{tab:FrequenzWerte}
  \begin{tabular}{c c c}
      \toprule
      $L$ / cm & $f$ / $\si{\mega\hertz}$ & $\upDelta f$ / $\si{\mega\hertz}$ \\
      \midrule
      50 & 304, 608, 675, 911, 1219 & $\approx$ 300 \\
      75 & 203, 405, 608, 806, 1009, 1211, 1414 & $\approx$ 200 \\
      100 & 150, 300, 454, 604, 758, 908, 1058, 1208, 1361 & $\approx$ 150 \\
      125 & 124, 244, 364, 484, 604, 724, 848, 968, 1088, 1208, 1328 & $\approx$ 125 \\
      \bottomrule
  \end{tabular}
\end{table}

\noindent
Die Anzahl der möglichen Moden ist abhängig von dem Bereich, in dem die Lichtverstärkung des Lasers stattfinden kann, was als Doppler-Breite $f=\SI{1500}{\mega\hertz}$ bezeichnet wird.
Die maximale Frequenz wurde dabei zwar nie direkt erreicht, kam aber immer sehr nah heran, weshalb anzunehmen ist, dass der Laser im Singlemode läuft.
Über die Gleichung
\begin{align}
  \upDelta f = \frac{1}{T} = \frac{c}{2L}
\end{align}
kann die Resonatorlänge $L$ durch die Schwebungsfrequenz $\upDelta f$ berechnet werden. In \autoref{tab:FrequenzRechnung} sind die bestimmten Resonatorlängen $L_\text{max}$ gegen die 
tatsächliche Resonatorlänge $L$ aufgetragen, um die Abweichung $\upDelta L$ zu bestimmen.

\begin{table}[H]
  \centering
  \caption{Experimentell bestimmte Werte der Resonatorlängen, sowie die Abweichung von der tatsächlichen Länge.}
  \label{tab:FrequenzRechnung}
  \begin{tabular}{c c c}
      \toprule
      $L$ / cm & $L_\text{max}$ / $\si{\centi\meter}$ & $\upDelta L$\\
      \midrule
      50 & 49.97 & 0.9993 $\%$ \\
      75 & 74.95 & 0.9993 $\%$ \\
      100 & 99.93 & 0.9993 $\%$ \\
      125 & 119.92 & 0.9593 $\%$ \\
      \bottomrule
  \end{tabular}
\end{table}