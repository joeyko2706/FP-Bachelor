\section{Diskussion}
\label{sec:Diskussion}
Die ermittelten Werte für die TEM-Moden weisen eine hohe Übereinstimmung mit den Theoriewerten auf, was an den Grafiken [\ref{fig:plot1}] und [\ref{fig:plot2}] zu erkennen ist.
So weisen die Moden deutlich erkennbare Maxima auf.
Auch die Polarisationsmessung zeigt eine hohe Genauigkeit mit der Ausgleichskurve und bestätigt somit die theoretische Vorhersage.
Die Betrachtung der Frequenzbreite konnte den Modenbetrieb des Lasers nachweisen und die Resonatorlängen auf eine Genauigkeit von $<1\%$ bestimmen. \newline
Mit einer Abweichung von $80.32\%$ der berechneten Wellenlänge $\lambda = \SI{508.436}{\nano\meter}$ vom Literaturwert $\lambda_\text{lit}=\SI{633}{\nano\meter}$ kann die Bestimmung der Wellenlänge
als nicht erfolgreich angesehen werden.
Die Betrachtung der Stabilitätsbedingung kann ebenfalls nicht als erfolgreich oder aussagekräftig eingestuft werden. Es konnten keine Resonatorlängen von über $\SI{150}{\centi\meter}$ eingestellt
werden, weil der Laser einfach nicht mehr stabilisiert werden konnte. Ebenso erfolglos konnte der Laser mit einer anderen Spiegelkonfiguration stabilisiert werden, egal welche Abstände die 
Spiegel dabei zueinander hatten. Die Messung der Stabilitätsbedingung ist demnach als fehlgeschlagen einzuordnen und kann die theoretischen Vorhersagen weder falsifiziern, noch verifizeren. \newline
Der Hauptgrund für dieses Fehlschlagen ist wahrscheinlich, dass die Strahlachse des Lasers nicht mit der optischen Achse der gesamten Vorrichtung übereinstimmt. Grund für die Annahme war,
dass auf den beiden planen Spiegeln der Laserstrahl bei einem stabilen Laser nicht durch die Mitte der beiden Spiegel ging. Die beiden konkaven Spiegel mussten also die Abweichung beider 
Achsen korrigieren, was die plan-konkave Spiegelvorrichtung nicht mehr korrigieren konnte und somit kein stabiler Laser zustande kam.
Ein weiterer Grund liegt wohl darin, dass der Brennpunkt des semipermeablen Spiegels nicht in der Glasröhre lag. Somit ergab sich eine halbmondförmige Rückkopplung des Spiegels, anstatt wider erwarten
eine punktförmige Rückkopplung.
