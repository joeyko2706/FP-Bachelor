\section{Diskussion}
\label{sec:Diskussion}

Wenn die lineare Regression in \autoref{fig:plot1} und \autoref{fig:plot2} betrachtet wird, dann ist anzumerken, dass es zum Teil große Abweichungen von der Regressionsgerade gibt.
Das ist auch an der Steigung zu erkennen, die für das erste Isotop ein Unsicherheit von $\SI{10}{\percent}$ und bei dem zweiten eine Unsicherheit von $\SI{14}{\percent}$ hat.
Da das Isotop $^{85}\text{Rb}$ in der Natur mit einem Anteil von $\SI{72}{\percent}$ deutlich häufiger vorkommt, folgt, dass der zweite Peak in den Resonanzen diesem Isotop 
zuzusprechen ist. Mit den jeweiligen Steigungen ergeben sich die Kernspins über die Landé-Faktoren zu den folgenden Werten,
\begin{align*}
    I_1 &= 2.05312 \pm 0.29526, \\
    I_2 &= 2.66466 \pm 0.45657.
\end{align*}
Die Literaturwerte \cite{RubidiumSpin} sind dabei
\begin{align*}
    I_{^{85}\text{Rb}} &= \sfrac52, \\
    I_{^{87}\text{Rb}} &= \sfrac32.
\end{align*}
Das entspricht einer Abweichung von $\SI{82.13\pm 11.81}{\percent}$ für $^{85}\text{Rb}$ und $\SI{177.64\pm 30.42}{\percent}$ für $^{87}\text{Rb}$. 
Das aus der Aufnahme des Oszilloskopes [\ref{fig:Oszilloskop}] bestimmte Isotopenverhältnis von  $\SI{40}{\percent}$ kommt mit einer Genauigkeit von $\SI{97.22}{\percent}$ an das in
der Natur vorkommende Verhältnis heran. \newline
Die für den Kernspin berechneten Werte weichen von den Literaturwerten zu stark ab, als das sie als signifikant einzustufen sind. Die Abweichungen der Isotopenverhältnisse ist mit $\SI{2.7}{\percent}$ hingegen als sehr gut einzuschätzen.
Ein Grund für die hohen Abweichungen kann in dem Aufbau des Strahlenganges liegen. Sind die Linsen nicht richtig eingestellt worden, zum Beispiel wenn die Linsen nicht im Brennpunkt sind, dann
lassen sich nur ungenaue Messwerte aufnehmen.
Ein weiterer Grund für die große Ungenauigkeit liegt vermutlich darin, dass bei Werten mit einer höheren Frequenz $f_{\text{RF}}$ auch die Peaks der Resonanzen nicht mehr gemeinsam
in der Reichweite des Sweep-Feldes zu erkennen waren. Es mussten deshalb zwei, voneinander unabhängige, horizontalen Magnetfelder eingestellt werden. Dies steigert die Unsicherheit, da
schneller Schwankungen auftreten können.
Des weiteren ist das hohe Schwanken des Betriebsstromes des horizontalen Magnetfeldes ein Grund für ungenaue Messwerte. Selbst wenn an dem Regler nichts verstellt wurde,
schwankte der Strom erheblich, sodass es zum Teil schwierig war die beiden Peaks vernünftig zu messen.