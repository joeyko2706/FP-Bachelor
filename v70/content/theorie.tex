\section{Ziel}
\label{sec:Ziel}

Ziel des Versuches ist es die Grundlagen der Vakuumphysik zu erlernen. Dafür wird eine Evakuierungskurve für zwei verschiedene Pumpartypen, einer Drehschieberpumpe
und einer Turbomolekularpumpe, durchgeführt. Es wird außerdem das effektive Saugvermögen beider Pumpen bestimmt.

\section{Theorie}
\label{sec:Theorie}

\subsection{Theorie des Vakuum}
\label{subsec:vakuum}

Als Vakuum wird der Zustand eines Gases bezeichnet, wenn der Druck innerhalb dieses Behälters unterhalb dem Druck außerhalb ist. Genauer wird das Vakuum als Druckbereich unterhalb
$\SI{300}{\milli\bar}$ bezeichnet, dem geringstmöglichen Druck auf der Erdoberfläche. Eine Möglichkeit Vakuen zu definieren ist über die mittlere freie Weglänge $\bar l$, die die
Strecke beschreibt, bis ein Teilchen im Mittel auf ein anderes trifft. Sie ist definiert als
\begin{align}
    \bar l = \frac{k_B T}{\sqrt 2 \pi\cdot p d_m^2},
\end{align}
wobei $d_m$ der Moleküldurchmesser, $p$ der Druck, $k_B$ die Boltzmann-Konstante und $T$ die Temperatur ist.
Das Vakuum ist dabei in verschiedene Teilbereiche einzuordnen, die in Abhängigkeit vom Druck
und der mittleren freien Weglänge $\bar l$, in \autoref{tab:Vakuen} aufgetragen sind.

\begin{table}[H]
    \centering
    \caption{Druckbereiche in der Vakuumtechnik \cite{EinfuehrungVakuum}.}
    \label{tab:Vakuen}
    \begin{tabular}{c c c}
        \toprule
        Druckbereich & Druck / hPa &  \\
        \midrule
        Atmosphärendruck    & $1.013,25$    & $6,8\cdot10^{-8}$   \\
        Grobvakuum (GV)     & $300 - 1$     & $10^{-8}-10^{-4}$   \\
        Feinvakuum (FV)     & $1 - 10^{-3}$ & $10^{-4} - 10^{-1}$ \\
        Hochvakuum (HV)     & $10^{-3} - 10^{-8}$ & $10^{-1} - 10^{4}$ \\
        Ultrahochvakuum (UHV) & $10^{-8} - 10^{-11}$ & $10^{4} - 10^{7}$ \\
        Extrem hohes Vakuum (XHV) & $<10^{-11}$ & $<10^{-7}$ \\
        \bottomrule
    \end{tabular}
\end{table}

Der Druck wird mithilfe der barometrischen Höhenformel beschrieben,
\begin{align}
    \label{eqn:baroFormel1}
    p_h = p_0\cdot \exp\left({-\frac{\rho_o gh}{p_0}}\right).
\end{align}
Dabei ist $p_h$ der Druck bei der Höhe $h$, $p_0$ der Atmosphärendruck auf Meereshöhe (\SI{1013,25}{\hecto\pascal}), $g$ die Erdbeschleunigung und $\rho_0$ die Dichte der Luft
auf Meereshöhe bei $\SI{0}{\celsius} = \SI{1,293}{\kilo\gram\per\meter\cubed}$. Die Formel (\ref{eqn:baroFormel1}) beschreibt den Druck als das Gewicht der über einer Fläche
stehenden Luftsäule. 
Wird angenommen, das die Dichte der Luft, die Erdbeschleunigung und der Atmosphärendruck auf Meereshöhe konstant sind, folgt eine vereinfachte Darstellung der barometrischen
Höhenformel,
\begin{align}
    \label{eqn:baroFormel2}
    p_h = p_0\cdot\exp\left(-\frac{h}{\SI{8,005}{\meter}}\right).
\end{align}

\subsection{Das ideale Gas}
\label{subsec:idGas}

Das Modell des idealen Gases ist eine vereinfachende Beschreibung für Gasprozesse. Die idealen Gasteilchen sind dabei frei und üben keine Anziehungs- oder Abstoßungskräfte 
aufeinander aus. Es finden lediglich elastische Stöße zwischen der Wand des Behälters, in dem sich das Gas befindet, und zwischen den Teilchen selber auf. Die Teilchen selber
belegen dabei aber kein Volumen und können sich auch nicht rotieren und nicht bewegen. Die kinetische Energie der Gasteilchen ist ausschließlich die translatorische Bewegung
im Raum. \newline
(Nach Quelle \cite{EinfuehrungVakuum}) Aus dem Boyle-Mariott'schen Gesetz $pV=const$ folgt, dass das Volumen einer Gasmenge bei konstanter Temperatur umgekehrt proportional zum Druck
ist. Gemeinsam mit dem Gay-Lussac'schen GEsetz $V=const\cdot T$ und der Stoffmenge, der allgemeinen Gaskonstanten und der Avogrado-Konstanten, die die Boltzmann-Konstante bildet,
folgt die allgemeine Gasgleichung,
\begin{align}
    \label{eqn:allgGasgl}
    pV=Nk_BT.
\end{align}
Dabei ist $p$ der Druck, $V$ das Volumen, der Teilchenzahl $N$ und $T$ die Temperatur.

\subsection{Verunreinigungen}
\label{subsec:verunreinigungen}

