\section{Auswertung}
\label{sec:Auswertung}

\subsection{Turbomolekularpumpe}
\label{subsec:Turbomolekularpumpe}

\subsubsection{Leckratenmessung}

Für die Leckratenmessung der Turbomolekularpumpe wurden für 4 Gleichgewichtsdrücke $p_1 \approx 5 \cdot 10^{-5} \: \si{mbar}$, $p_2 \approx 7 \cdot 10^{-5} \: \si{mbar}$,
$p_3 \approx 1 \cdot 10^{-4} \: \si{mbar}$ und $p_4 \approx 2 \cdot 10^{-4} \: \si{mbar}$ Messungen durchgeführt. Die Ergebnisse dieser Messungen sind in \autoref{tab:turboleck}
abgebildet.

\begin{table}[H]
  \centering
  \begin{tabular}{c|c|c|c|c}
    {$t \:/\: \si{s}$} & {$p_1 \:/\: \si{mbar} $} & {$p_2 \:/\: \si{mbar} $} & {$p_3 \:/\: \si{mbar} $} & {$p_4 \:/\: \si{mbar}$}\\
  \midrule
  0 & $(4,94 \pm \, 1,48)\cdot 10^{-5}$ & $(6,95 \pm \, 2,09)\cdot 10^{-5}$ & $(1,02 \pm \, 0,31)\cdot 10^{-4}$ & $(2,07 \pm \, 0,62)\cdot 10^{-4}$\\
  5 & $(1,44 \pm \, 0,43)\cdot 10^{-4}$ & $(2,11 \pm \, 0,63)\cdot 10^{-4}$ & $(2,52 \pm \, 0,76)\cdot 10^{-4}$ & $(6,41 \pm \, 1,92)\cdot 10^{-4}$\\
  10 & $(2,07 \pm \, 0,62)\cdot 10^{-4}$ & $(3,14 \pm \, 0,94)\cdot 10^{-4}$ & $(4,26 \pm \, 1,28)\cdot 10^{-4}$ & $(1,38 \pm \, 0,41)\cdot 10^{-3}$\\
  15 & $(2,81 \pm \, 0,84)\cdot 10^{-4}$ & $(4,08 \pm \, 1,22)\cdot 10^{-4}$ & $(6,08 \pm \, 1,82)\cdot 10^{-4}$ & $(2,51 \pm \, 0,75)\cdot 10^{-3}$\\
  20 & $(3,44 \pm \, 1,03)\cdot 10^{-4}$ & $(4,98 \pm \, 1,49)\cdot 10^{-4}$ & $(9,12 \pm \, 2,74)\cdot 10^{-4}$ & $(3,60 \pm \, 1,08)\cdot 10^{-3}$\\
  25 & $(4,05 \pm \, 1,22)\cdot 10^{-4}$ & $(6,72 \pm \, 2,02)\cdot 10^{-4}$ & $(1,21 \pm \, 0,36)\cdot 10^{-3}$ & $(4,99 \pm \, 1,50)\cdot 10^{-3}$\\
  30 & $(4,69 \pm \, 1,40)\cdot 10^{-4}$ & $(8,47 \pm \, 2,54)\cdot 10^{-4}$ & $(1,63 \pm \, 0,49)\cdot 10^{-3}$ & $(6,08 \pm \, 1,82)\cdot 10^{-3}$\\
  35 & $(5,63 \pm \, 1,69)\cdot 10^{-4}$ & $(1,07 \pm \, 0,32)\cdot 10^{-3}$ & $(2,09 \pm \, 0,63)\cdot 10^{-3}$ & $(8,01 \pm \, 2,40)\cdot 10^{-3}$\\
  40 & $(6,79 \pm \, 2,04)\cdot 10^{-4}$ & $(1,29 \pm \, 0,38)\cdot 10^{-3}$ & $(2,61 \pm \, 0,78)\cdot 10^{-3}$ & $(9,89 \pm \, 2,97)\cdot 10^{-3}$\\
  45 & $(7,94 \pm \, 2,38)\cdot 10^{-4}$ & $(1,51 \pm \, 0,45)\cdot 10^{-3}$ & $(3,07 \pm \, 0,92)\cdot 10^{-3}$ & $(1,07 \pm \, 0,32)\cdot 10^{-2}$\\
  50 & $(9,46 \pm \, 2,84)\cdot 10^{-4}$ & $(1,80 \pm \, 0,54)\cdot 10^{-3}$ & $(3,59 \pm \, 1,08)\cdot 10^{-3}$ & $(1,16 \pm \, 0,35)\cdot 10^{-2}$\\
  55 & $(1,07 \pm \, 0,32)\cdot 10^{-3}$ & $(2,10 \pm \, 0,63)\cdot 10^{-3}$ & $(4,22 \pm \, 1,27)\cdot 10^{-3}$ & $(1,25 \pm \, 0,38)\cdot 10^{-2}$\\
  60 & $(1,23 \pm \, 0,37)\cdot 10^{-3}$ & $(2,48 \pm \, 0,74)\cdot 10^{-3}$ & $(4,87 \pm \, 1,46)\cdot 10^{-3}$ & $(1,38 \pm \, 0,41)\cdot 10^{-2}$\\
  65 & $(1,39 \pm \, 0,42)\cdot 10^{-3}$ & $(2,77 \pm \, 0,83)\cdot 10^{-3}$ & $(5,13 \pm \, 1,54)\cdot 10^{-3}$ & $(1,48 \pm \, 0,44)\cdot 10^{-2}$\\
  70 & $(1,57 \pm \, 0,47)\cdot 10^{-3}$ & $(3,10 \pm \, 0,93)\cdot 10^{-3}$ & $(5,68 \pm \, 1,70)\cdot 10^{-3}$ & $(1,60 \pm \, 0,48)\cdot 10^{-2}$\\
  75 & $(1,73 \pm \, 0,52)\cdot 10^{-3}$ & $(3,45 \pm \, 1,04)\cdot 10^{-3}$ & $(6,25 \pm \, 1,88)\cdot 10^{-3}$ & $(1,75 \pm \, 0,53)\cdot 10^{-2}$\\
  80 & $(1,96 \pm \, 0,59)\cdot 10^{-3}$ & $(3,79 \pm \, 1,14)\cdot 10^{-3}$ & $(6,89 \pm \, 2,07)\cdot 10^{-3}$ & $(1,91 \pm \, 0,57)\cdot 10^{-2}$\\
  85 & $(2,18 \pm \, 0,65)\cdot 10^{-3}$ & $(4,23 \pm \, 1,27)\cdot 10^{-3}$ & $(7,57 \pm \, 2,27)\cdot 10^{-3}$ & $(2,02 \pm \, 0,61)\cdot 10^{-2}$\\
  90 & $(2,42 \pm \, 0,73)\cdot 10^{-3}$ & $(4,66 \pm \, 1,40)\cdot 10^{-3}$ & $(8,66 \pm \, 2,60)\cdot 10^{-3}$ & $(2,12 \pm \, 0,64)\cdot 10^{-2}$\\
  95 & $(2,63 \pm \, 0,79)\cdot 10^{-3}$ & $(4,96 \pm \, 1,49)\cdot 10^{-3}$ & $(9,24 \pm \, 2,77)\cdot 10^{-3}$ & $(2,22 \pm \, 0,67)\cdot 10^{-2}$\\
  100 & $(2,84 \pm \, 0,85)\cdot 10^{-3}$ & $(5,13 \pm \, 1,54)\cdot 10^{-3}$ & $(1,01 \pm \, 0,30)\cdot 10^{-2}$ & $(2,31 \pm \, 0,69)\cdot 10^{-2}$\\
  105 & $(3,09 \pm \, 0,93)\cdot 10^{-3}$ & $(5,42 \pm \, 1,63)\cdot 10^{-3}$ & $(1,05 \pm \, 0,32)\cdot 10^{-2}$ & $(2,44 \pm \, 0,76)\cdot 10^{-2}$\\
  110 & $(3,34 \pm \, 1,00)\cdot 10^{-3}$ & $(5,84 \pm \, 1,75)\cdot 10^{-3}$ & $(1,07 \pm \, 0,32)\cdot 10^{-2}$ & $(2,54 \pm \, 0,76)\cdot 10^{-2}$\\
  115 & $(3,57 \pm \, 1,07)\cdot 10^{-3}$ & $(6,13 \pm \, 1,84)\cdot 10^{-3}$ & $(1,11 \pm \, 0,33)\cdot 10^{-2}$ & $(2,65 \pm \, 0,79)\cdot 10^{-2}$\\
  120 & $(3,83 \pm \, 1,15)\cdot 10^{-3}$ & $(6,55 \pm \, 1,97)\cdot 10^{-3}$ & $(1,16 \pm \, 0,35)\cdot 10^{-2}$ & $(2,78 \pm \, 0,83)\cdot 10^{-2}$\\

  \end{tabular}
  \caption{Messwerte zur Leckratenmessung der Turbomolekularpumpe bei Gleichgewichtsdrücken $p_1=(4,94 \pm \, 1,48)\cdot 10^{-5} \, \si{mbar}$,
          $p_2=(6,95 \pm \, 2,09)\cdot 10^{-5} \, \si{mbar}$, $p_3=(1,02 \pm \, 0,31)\cdot 10^{-4} \, \si{mbar}$ und $p_4=(2,07 \pm \, 0,62)\cdot 10^{-4} \, \si{mbar}$,
          mit jeweils dem systematischen Fehler der Messgröße.}
  \label{tab:turboleck}
\end{table}

\noindent
Die Messwerte sind in \autoref{fig:plotturboleck1},\autoref{fig:plotturboleck2}, \autoref{fig:plotturboleck3} und \autoref{fig:plotturboleck4} graphisch dargestellt.


\begin{figure}[H]
    \centering
    \includegraphics[width=0.7\textwidth]{build/plotturboleck1.pdf}
    \caption{Grafische Darstellung der Messdaten von $p_1 = (4,94 \pm \, 1,48)\cdot 10^{-5} \: \si{mbar}$.}
    \label{fig:plotturboleck1}
\end{figure}

\begin{figure}[H]
  \centering
  \includegraphics[width=0.7\textwidth]{build/plotturboleck2.pdf}
  \caption{Grafische Darstellung der Messdaten von $p_2 = (6,95 \pm \, 2,09)\cdot 10^{-5} \: \si{mbar}$.}
  \label{fig:plotturboleck2}
\end{figure}

\begin{figure}[H]
  \centering
  \includegraphics[width=0.7\textwidth]{build/plotturboleck3.pdf}
  \caption{Grafische Darstellung der Messdaten von $p_3 = (1,02 \pm \, 0,31)\cdot 10^{-4} \: \si{mbar}$.}
  \label{fig:plotturboleck3}
\end{figure}

\begin{figure}[H]
\centering
\includegraphics[width=0.7\textwidth]{build/plotturboleck4.pdf}
\caption{Grafische Darstellung der Messdaten von $p_4 = (2,07 \pm \, 0,62)\cdot 10^{-4} \: \si{mbar}$.}
\label{fig:plotturboleck4}
\end{figure}

%%%%%%%%%%%%%%%%%%%%%%%%%%%%%%%%%%%%%%%%%%%%%%%%%%%%%%%%%%%%%%%%%%%%%%%%%%%%%%%%%%%%%%%%%%%%%%%%%%%%%%%%%%%%%%%%
\noindent
Es wurde jeweils eine lineare Ausgleichsrechnung der Form
\begin{equation}
	y(x)=mx+b
\end{equation}
durchgeführt. Über den Steigungsparameter $m$ kann nun jewels das Saugvermögen $S$, durch $S=\frac{V}{p_G}m$ berechnet werden. Dafür wird ein Volumen
von $V = (33 \pm \, 3,3 \: \si{\litre})$ angenommen.

\noindent
\autoref{fig:plotturboleck1} :
	\begin{align*}
		p_1=& (4,94 \pm \, 1,48)\cdot 10^{-5} \, \si{mbar}\\
		m_1=& (3,15 \pm \, 0,14)\cdot 10^{-5} \, \si{mbar/s}\\
		S_1=& \SI{21 \pm 7}{l/s}
	\end{align*}
\autoref{fig:plotturboleck2} :
	\begin{align*}
		p_2=& (6,95 \pm \, 2,09)\cdot 10^{-5} \, \si{mbar}\\
		m_2=& (5,66 \pm 0,18)\cdot 10^{-5} \, \si{mbar/s}\\
		S_2=& \SI{27 \pm 9}{l/s}
	\end{align*}
\autoref{fig:plotturboleck3} :
	\begin{align*}
		p_3=& (1,02 \pm \, 0,31)\cdot 10^{-4} \, \si{mbar}\\
		m_3=& (1,05 \pm \, 0,03)\cdot 10^{-4} \, \si{mbar/s}\\
		S_3=& \SI{34 \pm 14}{l/s}
	\end{align*}
\autoref{fig:plotturboleck4} :
	\begin{align*}
		p_4=& (2,07 \pm \, 0,62)\cdot 10^{-4} \, \si{mbar}\\
		m_4=& (2,38 \pm \, 0,02)\cdot 10^{-4} \, \si{mbar/s}\\
		S_4=& \SI{38 \pm 13}{l/s}
  \end{align*}




%%%%%%%%%%%%%%%%%%%%%%%%%%%%%%%%%%%%%%%%%%%%%%%%%%%%%%%%%%%%%%%%%%%%%%%%%%%%%%%%%%%%%%%%%%%%%%%%%%%%%%%%%%%%%%%%
\subsubsection{Evakuierungsmessung}

Die Messwerte der Evakuierungsmessung der Turbomolekularpumpe sind in \autoref{tab:turboevak} abgebildet. Die Messung wurde drei Mal mit einem Anfangsdruck
von $p \approx 5 \cdot 10^{-3} \: \si{mbar}$ durchgeführt und daraus der Mittelwert $p_{\text{m}}$ gebildet.

\begin{table}[H]
  \centering
  \begin{tabular}{c|c|c|c|c}
    {$t \:/\: \si{s}$} & {$p_1 \:/\: \si{mbar} $} & {$p_2 \:/\: \si{mbar} $} & {$p_3 \:/\: \si{mbar} $} & {$p_{\text{m}} \:/\: \si{mbar}$}\\
  \midrule
  0    & $(5,01 \pm \, 1,50)\cdot 10^{-3}$ & $(4,98 \pm \, 1,49)\cdot 10^{-3}$ & $(5,03 \pm \, 1,51)\cdot 10^{-3}$ & $(5,01 \pm \, 0,015)\cdot 10^{-3}$\\
  5    & $(1,65 \pm \, 0,50)\cdot 10^{-3}$ & $(3,47 \pm \, 1,04)\cdot 10^{-4}$ & $(1,76 \pm \, 0,53)\cdot 10^{-3}$ & $(1,25 \pm \, 0,46)\cdot 10^{-3}$\\
  6    & $(1,23 \pm \, 0,37)\cdot 10^{-3}$ & $(2,98 \pm \, 0,89)\cdot 10^{-4}$ & $(6,85 \pm \, 2,05)\cdot 10^{-4}$ & $(7,38 \pm \, 2,71)\cdot 10^{-4}$\\  
  7    & $(6,91 \pm \, 2,07)\cdot 10^{-4}$ & $(2,23 \pm \, 0,67)\cdot 10^{-4}$ & $(6,11 \pm \, 1,83)\cdot 10^{-4}$ & $(5,08 \pm \, 1,45)\cdot 10^{-4}$\\
  8    & $(4,37 \pm \, 1,31)\cdot 10^{-4}$ & $(1,56 \pm \, 0,47)\cdot 10^{-4}$ & $(3,72 \pm \, 1,12)\cdot 10^{-4}$ & $(3,22 \pm \, 0,85)\cdot 10^{-4}$\\
  9    & $(3,10 \pm \, 0,93)\cdot 10^{-4}$ & $(1,13 \pm \, 0,34)\cdot 10^{-4}$ & $(2,73 \pm \, 0,82)\cdot 10^{-4}$ & $(2,32 \pm \, 0,61)\cdot 10^{-4}$\\
  10   & $(2,67 \pm \, 0,80)\cdot 10^{-4}$ & $(8,79 \pm \, 2,64)\cdot 10^{-5}$ & $(2,03 \pm \, 0,61)\cdot 10^{-4}$ & $(1,86 \pm \, 0,53)\cdot 10^{-4}$\\
  11   & $(1,67 \pm \, 0,50)\cdot 10^{-4}$ & $(7,19 \pm \, 2,16)\cdot 10^{-5}$ & $(1,44 \pm \, 0,43)\cdot 10^{-4}$ & $(1,28 \pm \, 0,29)\cdot 10^{-4}$\\
  12   & $(1,29 \pm \, 0,39)\cdot 10^{-4}$ & $(5,75 \pm \, 1,73)\cdot 10^{-5}$ & $(1,05 \pm \, 0,32)\cdot 10^{-4}$ & $(9,72 \pm \, 2,10)\cdot 10^{-5}$\\
  13   & $(9,21 \pm \, 2,76)\cdot 10^{-5}$ & $(4,29 \pm \, 1,29)\cdot 10^{-5}$ & $(8,79 \pm \, 2,64)\cdot 10^{-5}$ & $(7,43 \pm \, 1,58)\cdot 10^{-5}$\\
  14   & $(8,58 \pm \, 2,57)\cdot 10^{-5}$ & $(3,63 \pm \, 1,09)\cdot 10^{-5}$ & $(6,33 \pm \, 1,90)\cdot 10^{-5}$ & $(6,18 \pm \, 1,43)\cdot 10^{-5}$\\
  15   & $(5,92 \pm \, 1,78)\cdot 10^{-5}$ & $(2,93 \pm \, 0,88)\cdot 10^{-5}$ & $(5,25 \pm \, 1,58)\cdot 10^{-5}$ & $(4,70 \pm \, 0,91)\cdot 10^{-5}$\\
  20   & $(2,47 \pm \, 0,74)\cdot 10^{-5}$ & $(1,49 \pm \, 0,45)\cdot 10^{-5}$ & $(2,06 \pm \, 0,62)\cdot 10^{-5}$ & $(2,01 \pm \, 0,29)\cdot 10^{-5}$\\
  25   & $(1,46 \pm \, 0,44)\cdot 10^{-5}$ & $(1,15 \pm \, 0,35)\cdot 10^{-5}$ & $(1,30 \pm \, 0,39)\cdot 10^{-5}$ & $(1,30 \pm \, 0,09)\cdot 10^{-5}$\\
  30   & $(1,18 \pm \, 0,35)\cdot 10^{-5}$ & $(1,01 \pm \, 0,30)\cdot 10^{-5}$ & $(1,08 \pm \, 0,32)\cdot 10^{-5}$ & $(1,09 \pm \, 0,05)\cdot 10^{-5}$\\
  35   & $(1,04 \pm \, 0,31)\cdot 10^{-5}$ & $(9,34 \pm \, 2,80)\cdot 10^{-6}$ & $(9,65 \pm \, 2,90)\cdot 10^{-6}$ & $(9,80 \pm \, 0,32)\cdot 10^{-6}$\\
  40   & $(9,63 \pm \, 2,89)\cdot 10^{-6}$ & $(8,65 \pm \, 2,60)\cdot 10^{-6}$ & $(8,90 \pm \, 2,67)\cdot 10^{-6}$ & $(9,06 \pm \, 0,30)\cdot 10^{-6}$\\
  45   & $(9,10 \pm \, 2,73)\cdot 10^{-6}$ & $(8,21 \pm \, 2,46)\cdot 10^{-6}$ & $(8,40 \pm \, 2,52)\cdot 10^{-6}$ & $(8,57 \pm \, 0,27)\cdot 10^{-6}$\\
  50   & $(8,59 \pm \, 2,58)\cdot 10^{-6}$ & $(7,78 \pm \, 2,33)\cdot 10^{-6}$ & $(7,92 \pm \, 2,38)\cdot 10^{-6}$ & $(8,10 \pm \, 0,25)\cdot 10^{-6}$\\
  55   & $(8,17 \pm \, 2,45)\cdot 10^{-6}$ & $(7,50 \pm \, 2,25)\cdot 10^{-6}$ & $(7,55 \pm \, 2,27)\cdot 10^{-6}$ & $(7,74 \pm \, 0,22)\cdot 10^{-6}$\\
  60   & $(7,80 \pm \, 2,34)\cdot 10^{-6}$ & $(7,21 \pm \, 2,13)\cdot 10^{-6}$ & $(7,23 \pm \, 2,17)\cdot 10^{-6}$ & $(7,41 \pm \, 0,20)\cdot 10^{-6}$\\
  65   & $(7,52 \pm \, 2,26)\cdot 10^{-6}$ & $(6,97 \pm \, 2,09)\cdot 10^{-6}$ & $(6,98 \pm \, 2,09)\cdot 10^{-6}$ & $(7,16 \pm \, 0,19)\cdot 10^{-6}$\\
  70   & $(7,27 \pm \, 2,18)\cdot 10^{-6}$ & $(6,79 \pm \, 2,04)\cdot 10^{-6}$ & $(6,78 \pm \, 2,03)\cdot 10^{-6}$ & $(6,95 \pm \, 0,17)\cdot 10^{-6}$\\
  75   & $(7,05 \pm \, 2,12)\cdot 10^{-6}$ & $(6,61 \pm \, 1,98)\cdot 10^{-6}$ & $(6,55 \pm \, 1,97)\cdot 10^{-6}$ & $(6,74 \pm \, 0,16)\cdot 10^{-6}$\\
  80   & $(6,89 \pm \, 2,07)\cdot 10^{-6}$ & $(6,43 \pm \, 1,93)\cdot 10^{-6}$ & $(6,40 \pm \, 1,92)\cdot 10^{-6}$ & $(6,57 \pm \, 0,16)\cdot 10^{-6}$\\
  85   & $(6,71 \pm \, 2,01)\cdot 10^{-6}$ & $(6,27 \pm \, 1,88)\cdot 10^{-6}$ & $(6,23 \pm \, 1,87)\cdot 10^{-6}$ & $(6,40 \pm \, 0,16)\cdot 10^{-6}$\\
  90   & $(6,57 \pm \, 1,97)\cdot 10^{-6}$ & $(6,16 \pm \, 1,85)\cdot 10^{-6}$ & $(6,12 \pm \, 1,84)\cdot 10^{-6}$ & $(6,28 \pm \, 0,15)\cdot 10^{-6}$\\
  95   & $(6,40 \pm \, 1,92)\cdot 10^{-6}$ & $(6,02 \pm \, 1,81)\cdot 10^{-6}$ & $(6,00 \pm \, 1,80)\cdot 10^{-6}$ & $(6,14 \pm \, 0,13)\cdot 10^{-6}$\\
  100  & $(6,28 \pm \, 1,88)\cdot 10^{-6}$ & $(5,93 \pm \, 1,78)\cdot 10^{-6}$ & $(5,87 \pm \, 1,76)\cdot 10^{-6}$ & $(6,03 \pm \, 0,13)\cdot 10^{-6}$\\
  105  & $(6,15 \pm \, 1,85)\cdot 10^{-6}$ & $(5,83 \pm \, 1,75)\cdot 10^{-6}$ & $(5,76 \pm \, 1,73)\cdot 10^{-6}$ & $(5,91 \pm \, 0,12)\cdot 10^{-6}$\\
  110  & $(6,06 \pm \, 1,82)\cdot 10^{-6}$ & $(5,74 \pm \, 1,72)\cdot 10^{-6}$ & $(5,67 \pm \, 1,70)\cdot 10^{-6}$ & $(5,82 \pm \, 0,12)\cdot 10^{-6}$\\
  115  & $(5,99 \pm \, 1,80)\cdot 10^{-6}$ & $(5,65 \pm \, 1,70)\cdot 10^{-6}$ & $(5,59 \pm \, 1,68)\cdot 10^{-6}$ & $(5,74 \pm \, 0,13)\cdot 10^{-6}$\\
  120  & $(5,87 \pm \, 1,76)\cdot 10^{-6}$ & $(5,59 \pm \, 1,68)\cdot 10^{-6}$ & $(5,51 \pm \, 1,65)\cdot 10^{-6}$ & $(5,66 \pm \, 0,11)\cdot 10^{-6}$\\
  \end{tabular}
  \caption{Messwerte für die Evakuierungskurve der Turbomolekularpumpe mit jeweils dem systematischen Fehler der Messgröße und dem statistischen Fehler
            des Mittelwerts.}
  \label{tab:turboevak}
\end{table}
\noindent
Da der statistische Fehler des Mittelwerts klein gegen den systematischen Fehler ist, wird im Folgenden lediglich der systematische Fehler berücksichtigt.
Die Daten der Mittelwertsmessung sind in \autoref{tab:ln1} als $\ln\left(\frac{p(t) - p_E}{p_0 - p_E}\right)$ abgebildet.
Dabei sind $p_0 = (5,01 \pm \, 1,50)\cdot 10^{-3} \: \si{mbar}$ und $p_\text{E} = (4,2 \pm \, 1,26)\cdot 10^{-6} \: \si{mbar}$.

\begin{table}[H]
  \centering
  \begin{tabular}{c|c}
    $t/s$ & $ln(\frac{p(t)-p_0}{p_0 - p_E})$ \\
    \midrule
    0     & $ 0,0 $ \\
    5     & $ -1,39  \pm \,  0,42 $ \\
    6     & $ -1,92  \pm \,  0,43 $ \\
    7     & $ -2,3   \pm \,  0,43 $ \\
    8     & $ -2,76  \pm \,  0,43 $ \\
    9     & $ -3,09  \pm \,  0,43 $ \\
    10    & $ -3,32  \pm \,  0,43 $ \\
    11    & $ -3,7   \pm \,  0,43 $ \\
    12    & $ -3,99  \pm \,  0,43 $ \\
    13    & $ -4,27  \pm \,  0,44 $ \\
    14    & $ -4,46  \pm \,  0,44 $ \\
    15    & $ -4,76  \pm \,  0,45 $ \\
    20    & $ -5,75  \pm \,  0,49 $ \\
    25    & $ -6,34  \pm \,  0,55 $ \\
    30    & $ -6,62  \pm \,  0,6 $ \\
    35    & $ -6,8   \pm \,  0,65 $ \\
    40    & $ -6,94  \pm \,  0,69 $ \\
    45    & $ -7,04  \pm \,  0,72 $ \\
    50    & $ -7,16  \pm \,  0,76 $ \\
    55    & $ -7,25  \pm \,  0,8 $ \\
    60    & $ -7,35  \pm \,  0,85 $ \\
    65    & $ -7,43  \pm \,  0,89 $ \\
    70    & $ -7,51  \pm \,  0,94 $ \\
    75    & $ -7,59  \pm \,  0,99 $ \\
    80    & $ -7,65  \pm \,  1,03 $ \\
    85    & $ -7,73  \pm \,  1,08 $ \\
    90    & $ -7,78  \pm \,  1,13 $ \\
    95    & $ -7,86  \pm \,  1,19 $ \\
    100   & $ -7,92  \pm \,  1,24 $ \\
    105   & $ -7,98  \pm \,  1,3 $ \\
    110   & $ -8,03  \pm \,  1,36 $ \\
    115   & $ -8,08  \pm \,  1,42 $ \\
    120   & $ -8,14  \pm \,  1,48 $ \\
    \bottomrule
  \end{tabular}
  \caption{$ln(\frac{p(t)-p_0}{p_0 - p_E})$ bestimmt aus den Messwerten der Evakuierungsmessung.}
  \label{tab:ln1}
\end{table}
\noindent
Sie sind ebenfalls in \autoref{fig:plotturboevak} graphisch dargestellt.

\begin{figure}[H]
  \centering
  \includegraphics[width=0.7\textwidth]{build/plotturboevak.pdf}
  \caption{Logarithmische Darstellung der Evakuierungskurve der Turbomolekularpumpe.}
  \label{fig:plotturboevak}
\end{figure}
\noindent
Es wurden erneut lineare Ausgleichsrechnungen der Form
\begin{equation}
	y(x)=mx+b
\end{equation}
durchgeführt, wobei die Messdaten in zwei Bereiche mit unterschiedlichen Steigungen unterteilt wurden. 
Aus den Regressionsparametern $m$ können dann die Saugvermögen in den Bereichen über $S=-mV$ bestimmt werden.

\noindent
Bereich 1: $5 \cdot 10^{-3} \, \si{mbar} \geq p \geq 2 \cdot 10^{-5} \, \si{mbar}$\\
\begin{align*}
	m_1=& -\SI{0,302 \pm 0,012}{1/s}\\
	S_1=& \SI{10,0 \pm 1,1}{l/s}
\end{align*}
Bereich 2: $2 \cdot 10^{-5} \, \si{mbar} \geq p \geq 5 \cdot 10^{-6} \, \si{mbar}$\\
\begin{align*}
	m_2=& -\SI{0,018 \pm 0,001}{1/s}\\
	S_2=& \SI{0,59 \pm 0,07}{l/s}
\end{align*}


\subsection{Drehschieberpumpe}
\label{subsec:Drehschieberpumpee}


\subsubsection{Leckratenmessung}


Für die Leckratenmessung der Drehschieberpumpe wurden ebenfalls 4 Gleichgewichtsdrücke vermessen.
Für den ersten Wert von $p_1 \approx 0,5 \: \si{mbar}$ wurde die Messung drei mal durchgeführt und daraus der Mittelwert gebildet.
Die Ergebnisse dieser Messungen sind in \autoref{tab:drehleck1} abgebildet.


\begin{table}[H]
  \centering
  \begin{tabular}{c|c|c|c|c}
    {$t \:/\: \si{s}$} & {$p_1 \:/\: \si{mbar} $} & {$p_2 \:/\: \si{mbar} $} & {$p_3 \:/\: \si{mbar} $} & {$p_{\text{m}} \:/\: \si{mbar}$}\\
  \midrule
  0     & $ 0,5  \pm \,  0,05 $ & $ 0,5  \pm \,  0,05 $ & $ 0,5  \pm \,  0,05 $ & $ 0,5   \pm \,  0,01 $ \\
  10    & $ 1,9  \pm \,  0,19 $ & $ 1,7  \pm \,  0,17 $ & $ 1,7  \pm \,  0,17 $ & $ 1,77  \pm \,  0,07 $ \\
  20    & $ 2,0  \pm \,  0,20 $ & $ 1,9  \pm \,  0.19 $ & $ 1,8  \pm \,  0,18 $ & $ 1,9   \pm \,  0,06 $ \\
  30    & $ 2,1  \pm \,  0,21 $ & $ 2,0  \pm \,  0,20 $ & $ 1,8  \pm \,  0,18 $ & $ 1,97  \pm \,  0,09 $ \\
  40    & $ 2,2  \pm \,  0,22 $ & $ 2,1  \pm \,  0,21 $ & $ 1,9  \pm \,  0,19 $ & $ 2,07  \pm \,  0,09 $ \\
  50    & $ 2,4  \pm \,  0,24 $ & $ 2,2  \pm \,  0,22 $ & $ 2,0  \pm \,  0,20 $ & $ 2,2   \pm \,  0,12 $ \\
  60    & $ 2,5  \pm \,  0,25 $ & $ 2,3  \pm \,  0,23 $ & $ 2,0  \pm \,  0,20 $ & $ 2,27  \pm \,  0,15 $ \\
  70    & $ 2,7  \pm \,  0,27 $ & $ 2,4  \pm \,  0,24 $ & $ 2,1  \pm \,  0,21 $ & $ 2,4   \pm \,  0,17 $ \\
  80    & $ 2,8  \pm \,  0,28 $ & $ 2,6  \pm \,  0,26 $ & $ 2,1  \pm \,  0,21 $ & $ 2,5   \pm \,  0,21 $ \\
  90    & $ 2,9  \pm \,  0,29 $ & $ 2,7  \pm \,  0,27 $ & $ 2,2  \pm \,  0,22 $ & $ 2,6   \pm \,  0,21 $ \\
  100   & $ 3,0  \pm \,  0,30 $ & $ 2,8  \pm \,  0,28 $ & $ 2,3  \pm \,  0,23 $ & $ 2,7   \pm \,  0,21 $ \\
  110   & $ 3,1  \pm \,  0,31 $ & $ 2,9  \pm \,  0,29 $ & $ 2,3  \pm \,  0,23 $ & $ 2,77  \pm \,  0,24 $ \\
  120   & $ 3,3  \pm \,  0,33 $ & $ 3,0  \pm \,  0,30 $ & $ 2,4  \pm \,  0,24 $ & $ 2,9   \pm \,  0,26 $ \\
  130   & $ 3,4  \pm \,  0,34 $ & $ 3,1  \pm \,  0,31 $ & $ 2,5  \pm \,  0,25 $ & $ 3,0   \pm \,  0,26 $ \\
  140   & $ 3,6  \pm \,  0,36 $ & $ 3,2  \pm \,  0,32 $ & $ 2,6  \pm \,  0,26 $ & $ 3,13  \pm \,  0,29 $ \\
  150   & $ 3,7  \pm \,  0,37 $ & $ 3,3  \pm \,  0,33 $ & $ 2,7  \pm \,  0,27 $ & $ 3,23  \pm \,  0,29 $ \\
  160   & $ 3,9  \pm \,  0,39 $ & $ 3,5  \pm \,  0,35 $ & $ 2,7  \pm \,  0,27 $ & $ 3,37  \pm \,  0,35 $ \\
  170   & $ 4,0  \pm \,  0,40 $ & $ 3,6  \pm \,  0,36 $ & $ 2,8  \pm \,  0,28 $ & $ 3,47  \pm \,  0,35 $ \\
  180   & $ 4,1  \pm \,  0,41 $ & $ 3,7  \pm \,  0,37 $ & $ 2,9  \pm \,  0,29 $ & $ 3,57  \pm \,  0,35 $ \\
  190   & $ 4,3  \pm \,  0,43 $ & $ 3,8  \pm \,  0,38 $ & $ 2,9  \pm \,  0,29 $ & $ 3,67  \pm \,  0,41 $ \\
  200   & $ 4,4  \pm \,  0,44 $ & $ 3,9  \pm \,  0,39 $ & $ 3,0  \pm \,  0,30 $ & $ 3,77  \pm \,  0,41 $ \\
  \end{tabular}
  \caption{Messwerte für die Leckratenmessung der Drehschieberpumpe mit jeweils dem systematischen Fehler der Messgröße und dem statistischen Fehler
           des Mittelwerts.}
  \label{tab:drehleck1}
\end{table}
\noindent
Da der statistische Fehler erneut klein gegen den systematischen Fehler ist, wird lediglich der systematische Fehler berücksichtigt.
Die Werte des Mittelwerts sind in \autoref{fig:plotdrehleck1} graphisch dargestellt.

\begin{figure}[H]
  \centering
  \includegraphics[width=0.7\textwidth]{build/plotdrehleck1.pdf}
  \caption{Grafische Darstellung der Messdaten von $p_{\text{m}}  = (0,5 \pm \, 0,05) \: \si{mbar}$.}
  \label{fig:plotdrehleck1}
\end{figure}
\noindent
Identisch zur Turbomolekularpumpe kann über die Regressionsparameter $m$ das Saugvermögen bestimmt werden.

\noindent
\autoref{fig:plotdrehleck1} :
	\begin{align*}
		p_{\text{m}} =& (0,50 \pm \, 0,15) \, \si{mbar}\\
		m_{\text{m}} =& (0,012 \pm \, 0,001) \, \si{mbar/s}\\
		S_{\text{m}} =& \SI{0,82 \pm 0,27}{l/s}
  \end{align*}

\noindent
Die restlichen Messwerte für $p_1 \approx 10 \: \si{mbar}$, $p_2 \approx 50 \: \si{mbar}$ und $p_3 \approx 100 \: \si{mbar}$ sind in
\autoref{tab:drehleck2} zu finden.


\begin{table}[H]
  \centering
  \begin{tabular}{c|c|c|c}
    {$t \:/\: \si{s}$} & {$p_1 \:/\: \si{mbar} $} & {$p_2 \:/\: \si{mbar} $} & {$p_3 \:/\: \si{mbar} $}\\
  \midrule
  0     & $ 9,9   \pm \,  4 $ & $ 50,3   \pm \,  4 $ & $ 99,7   \pm \,  4 $ \\ 
  10    & $ 32,0  \pm \,  4 $ & $ 76,3   \pm \,  4 $ & $ 143,8  \pm \,  4 $ \\
  20    & $ 35,0  \pm \,  4 $ & $ 94,2   \pm \,  4 $ & $ 181,7  \pm \,  4 $ \\
  30    & $ 38,0  \pm \,  4 $ & $ 112,2  \pm \,  4 $ & $ 214,7  \pm \,  4 $ \\
  40    & $ 41,0  \pm \,  4 $ & $ 130,0  \pm \,  4 $ & $ 249,0  \pm \,  4 $ \\
  50    & $ 43,0  \pm \,  4 $ & $ 147,9  \pm \,  4 $ & $ 283,4  \pm \,  4 $ \\
  60    & $ 45,0  \pm \,  4 $ & $ 165,8  \pm \,  4 $ & $ 317,7  \pm \,  4 $ \\
  70    & $ 48,0  \pm \,  4 $ & $ 183,7  \pm \,  4 $ & $ 351,8  \pm \,  4 $ \\
  80    & $ 51,0  \pm \,  4 $ & $ 202,5  \pm \,  4 $ & $ 382,5  \pm \,  4 $ \\
  90    & $ 55,0  \pm \,  4 $ & $ 220,4  \pm \,  4 $ & $ 416,5  \pm \,  4 $ \\
  100   & $ 57,0  \pm \,  4 $ & $ 238,2  \pm \,  4 $ & $ 450,4  \pm \,  4 $ \\
  110   & $ 60,0  \pm \,  4 $ & $ 256,1  \pm \,  4 $ & $ 487,2  \pm \,  4 $ \\
  120   & $ 62,0  \pm \,  4 $ & $ 274,0  \pm \,  4 $ & $ 520,3  \pm \,  4 $ \\
  130   & $ 64,0  \pm \,  4 $ & $ 291,9  \pm \,  4 $ & $ 553,1  \pm \,  4 $ \\
  140   & $ 66,0  \pm \,  4 $ & $ 309,7  \pm \,  4 $ & $ 585,5  \pm \,  4 $ \\
  150   & $ 68,0  \pm \,  4 $ & $ 327,5  \pm \,  4 $ & $ 617,1  \pm \,  4 $ \\
  160   & $ 72,0  \pm \,  4 $ & $ 345,5  \pm \,  4 $ & $ 648,0  \pm \,  4 $ \\
  170   & $ 75,0  \pm \,  4 $ & $ 363,3  \pm \,  4 $ & $ 678,0  \pm \,  4 $ \\
  180   & $ 78,0  \pm \,  4 $ & $ 381,2  \pm \,  4 $ & $ 707,1  \pm \,  4 $ \\
  190   & $ 80,0  \pm \,  4 $ & $ 399,0  \pm \,  4 $ & $ 735,2  \pm \,  4 $ \\
  200   & $ 82,0  \pm \,  4 $ & $ 416,6  \pm \,  4 $ & $ 762,2  \pm \,  4 $ \\
  \end{tabular}
  \caption{Messwerte für die Leckratenmessung der Drehschieberpumpe mit jeweils dem systematischen Fehler.}
  \label{tab:drehleck2}
\end{table}

\noindent
Ebenfalls Identisch zur Turbomolekularpumpe sind die restlichen Messwerte in \autoref{fig:plotdrehleck2}, \autoref{fig:plotdrehleck3}
und \autoref{fig:plotdrehleck4} graphisch dargestellt.

\begin{figure}[H]
  \centering
  \includegraphics[width=0.7\textwidth]{build/plotdrehleck2.pdf}
  \caption{Grafische Darstellung der Messdaten von $p_1 = (9,9 \pm \, 4) \: \si{mbar}$.}
  \label{fig:plotdrehleck2}
\end{figure}

\begin{figure}[H]
  \centering
  \includegraphics[width=0.7\textwidth]{build/plotdrehleck3.pdf}
  \caption{Grafische Darstellung der Messdaten von $p_2 = (50,3 \pm \, 4) \: \si{mbar}$.}
  \label{fig:plotdrehleck3}
\end{figure}

\begin{figure}[H]
  \centering
  \includegraphics[width=0.7\textwidth]{build/plotdrehleck4.pdf}
  \caption{Grafische Darstellung der Messdaten von $p_3 = (99,7 \pm \, 4) \: \si{mbar}$.}
  \label{fig:plotdrehleck4}
\end{figure}

\noindent
Die Steigungsparameter $m$ werden erneut genutzt, um über $S=\frac{V}{p_G}m$ das Saugvermögen $S$ zu berechnen.
Hierbei wird allerdings ein Volumen von $V = (34 \pm \, 3,4 \: \si{\litre})$ angenommen.

\noindent
\autoref{fig:plotdrehleck2} :
	\begin{align*}
		p_1 =& (9,9 \pm \, 4) \, \si{mbar}\\
		m_1 =& (0,289 \pm \, 0,015) \, \si{mbar/s}\\
		S_1 =& \SI{0,99 \pm 0,40}{l/s}
\end{align*}
\autoref{fig:plotdrehleck3} :
	\begin{align*}
		p_2 =& (50,3 \pm \, 4) \, \si{mbar}\\
		m_2 =& (1,804 \pm \, 0,006) \, \si{mbar/s}\\
		S_2 =& \SI{1,22 \pm 0,16}{l/s}
\end{align*}
\autoref{fig:plotdrehleck4} :
	\begin{align*}
		p_3 =& (99,7 \pm \, 4) \, \si{mbar}\\
		m_3 =& (3,309 \pm \, 0,025) \, \si{mbar/s}\\
		S_3 =& \SI{1,13 \pm 0,12}{l/s}
\end{align*}





\subsubsection{Evakuierungsmessung}

Die Messwerte für die Evakuierungsmessung der Drehschieberpumpe sind in \autoref{tab:drehevak} abgebildet.

\begin{table}[H]
  \centering
  \begin{tabular}{c|c||c|c}
    {$t \:/\: \si{s}$} & {$p \:/\: \si{mbar} $} & {$t \:/\: \si{s}$} & {$p \:/\: \si{mbar} $}\\
  \midrule
  0     & $ 1012,0  \pm \,  4 $ &  300  & $ 0,67  \pm \,  0,07 $ \\ 
  10    & $ 636,3  \pm \,  4 $  &  310  & $ 0,63  \pm \,  0,07 $ \\
  20    & $ 474,6  \pm \,  4 $  &  320  & $ 0,59  \pm \,  0,06 $ \\
  30    & $ 356,4  \pm \,  4 $  &  330  & $ 0,54  \pm \,  0,06 $ \\
  40    & $ 266,5  \pm \,  4 $  &  340  & $ 0,51  \pm \,  0,06 $ \\
  50    & $ 195,5  \pm \,  4 $  &  350  & $ 0,48  \pm \,  0,05 $ \\
  60    & $ 142,3  \pm \,  4 $  &  360  & $ 0,45  \pm \,  0,05 $ \\
  70    & $ 103,8  \pm \,  4 $  &  370  & $ 0,42  \pm \,  0,05 $ \\
  80    & $ 76,3  \pm \,  4 $    &  380  & $ 0,40  \pm \,  0,04 $ \\
  90    & $ 55,2  \pm \,  4 $    &  390  & $ 0,38  \pm \,  0,04 $ \\
  100   & $ 40,1  \pm \,  4 $    &  400  & $ 0,36  \pm \,  0,04 $ \\
  110   & $ 28,6  \pm \,  4 $    &  410  & $ 0,34  \pm \,  0,04 $ \\
  120   & $ 20,8  \pm \,  4 $    &   420  & $ 0,33  \pm \,  0,04 $ \\
  130   & $ 14,9  \pm \,  4 $       &  430  & $ 0,3  \pm \,  0,03 $ \\
  140   & $ 11,5  \pm \,  4 $       &  440  & $ 0,29  \pm \,  0,03 $ \\
  150   & $ 8,9   \pm \,  0,89 $    &  450  & $ 0,28  \pm \,  0,03 $ \\
  160   & $ 6,6   \pm \,  0,66 $    &  460  & $ 0,27  \pm \,  0,03 $ \\
  170   & $ 5,0   \pm \,  0,50 $     &  470  & $ 0,26  \pm \,  0,03 $ \\
  180   & $ 3,9   \pm \,  0,39 $    &  480  & $ 0,24  \pm \,  0,03 $ \\
  190   & $ 3,0   \pm \,  0,30 $     &  490  & $ 0,23  \pm \,  0,03 $ \\
  200   & $ 2,5   \pm \,  0,25 $    &  500  & $ 0,22  \pm \,  0,03 $ \\
  210   & $ 2,0   \pm \,  0,20 $     &  510  & $ 0,22  \pm \,  0,03 $ \\
  220   & $ 1,7   \pm \,  0,17 $    &  520  & $ 0,21  \pm \,  0,03 $ \\
  230   & $ 1,5   \pm \,  0,15 $    &  530  & $ 0,20  \pm \,  0,02 $ \\
  240   & $ 1,3   \pm \,  0,13 $    &  540  & $ 0,20  \pm \,  0,02 $ \\
  250   & $ 1,1   \pm \,  0,11 $    &  550  & $ 0,19  \pm \,  0,02 $ \\
  260   & $ 0,99  \pm \,  0,10 $     &  560  & $ 0,19  \pm \,  0,02 $ \\
  270   & $ 0,90   \pm \,  0,09 $    &  570  & $ 0,18  \pm \,  0,02 $ \\
  280   & $ 0,82  \pm \,  0,09 $    &  580  & $ 0,17  \pm \,  0,02 $ \\
  290   & $ 0,75  \pm \,  0,08 $    &  590  & $ 0,17  \pm \,  0,02 $ \\
  300   & $ 0,67  \pm \,  0,07 $    &  600  & $ 0,16  \pm \,  0,02 $ \\
  \end{tabular}
  \caption{Messwerte für die Evakuierungskurve der Drehschieberpumpe mit jeweils dem systematischen Fehler.}
  \label{tab:drehevak}
\end{table}

\noindent
Die Messwerte sind erneut in \autoref{tab:ln2} als $\ln\left(\frac{p(t) - p_E}{p_0 - p_E}\right)$ abgebildet. Dabei ist nun
$p_0 = (1012,0 \pm \, 4) \: \si{mbar}$ und $p_\text{E} = (1,6 \pm \, 0,16)\cdot 10^{-2} \: \si{mbar}$.

\begin{table}[H]
  \centering
  \begin{tabular}{c|c||c|c}
    {$t \:/\: \si{s}$} & { $\ln\left(\frac{p(t) - p_E}{p_0 - p_E}\right)$ } & {$t \:/\: \si{s}$} & {$\ln\left(\frac{p(t) - p_E}{p_0 - p_E}\right)$}\\
  \midrule
  0    & $ 0.0                  $ &  300  & $ -7,34  \pm \,  0,103 $ \\ 
  10   & $ -0,46  \pm \,  0,008 $ &  310  & $ -7,41  \pm \,  0,103 $ \\
  20   & $ -0,76  \pm \,  0,010 $ &  320  & $ -7,47  \pm \,  0,103 $ \\
  30   & $ -1,04  \pm \,  0,012 $ &  330  & $ -7,57  \pm \,  0,104 $ \\
  40   & $ -1,33  \pm \,  0,016 $ &  340  & $ -7,62  \pm \,  0,104 $ \\
  50   & $ -1,64  \pm \,  0,021 $ &  350  & $ -7,69  \pm \,  0,104 $ \\
  60   & $ -1,96  \pm \,  0,029 $ &  360  & $ -7,75  \pm \,  0,104 $ \\
  70   & $ -2,28  \pm \,  0,039 $ &  370  & $ -7,83  \pm \,  0,105 $ \\
  80   & $ -2,59  \pm \,  0,053 $ &  380  & $ -7,88  \pm \,  0,105 $ \\
  90   & $ -2,91  \pm \,  0,073 $ &  390  & $ -7,93  \pm \,  0,105 $ \\
  100  & $ -3,23  \pm \,  0,100 $ &  400  & $ -7,99  \pm \,  0,105 $ \\
  110  & $ -3,57  \pm \,  0,140 $ &  410  & $ -8,05  \pm \,  0,106 $ \\
  120  & $ -3,89  \pm \,  0,193 $ &  420  & $ -8,08  \pm \,  0,106 $ \\
  130  & $ -4,22  \pm \,  0,269 $ &  430  & $ -8,18  \pm \,  0,106 $ \\
  140  & $ -4,48  \pm \,  0,349 $ &  440  & $ -8,21  \pm \,  0,107 $ \\
  150  & $ -4,74  \pm \,  0,101 $ &  450  & $ -8,25  \pm \,  0,107 $ \\
  160  & $ -5,04  \pm \,  0,101 $ &  460  & $ -8,29  \pm \,  0,107 $ \\
  170  & $ -5,31  \pm \,  0,101 $ &  470  & $ -8,33  \pm \,  0,107 $ \\
  180  & $ -5,56  \pm \,  0,101 $ &  480  & $ -8,42  \pm \,  0,108 $ \\
  190  & $ -5,83  \pm \,  0,101 $ &  490  & $ -8,46  \pm \,  0,108 $ \\
  200  & $ -6,01  \pm \,  0,101 $ &  500  & $ -8,51  \pm \,  0,109 $ \\
  210  & $ -6,23  \pm \,  0,101 $ &  510  & $ -8,51  \pm \,  0,109 $ \\
  220  & $ -6,40  \pm \,  0,102 $ &  520  & $ -8,56  \pm \,  0,109 $ \\
  230  & $ -6,52  \pm \,  0,102 $ &  530  & $ -8,61  \pm \,  0,110 $ \\
  240  & $ -6,67  \pm \,  0,102 $ &  540  & $ -8,61  \pm \,  0,110 $ \\
  250  & $ -6,84  \pm \,  0,102 $ &  550  & $ -8,67  \pm \,  0,110 $ \\
  260  & $ -6,95  \pm \,  0,102 $ &  560  & $ -8,67  \pm \,  0,110 $ \\
  270  & $ -7,04  \pm \,  0,102 $ &  570  & $ -8,73  \pm \,  0,111 $ \\
  280  & $ -7,14  \pm \,  0,103 $ &  580  & $ -8,79  \pm \,  0,111 $ \\
  290  & $ -7,23  \pm \,  0,103 $ &  590  & $ -8,79  \pm \,  0,111 $ \\
  300  & $ -7,34  \pm \,  0,103 $ &  600  & $ -8,86  \pm \,  0,112 $ \\
  \end{tabular}
  \caption{$ln(\frac{p(t)-p_0}{p_0 - p_E})$ bestimmt aus den Messwerten der Evakuierungsmessung.}
  \label{tab:ln2}
\end{table}
\noindent

Die Werte aus \autoref{tab:ln2} sind ebenfalls in \autoref{fig:plotdrehevak} graphisch dargestellt.

\begin{figure}[H]
  \centering
  \includegraphics[width=0.7\textwidth]{build/plotdrehevak.pdf}
  \caption{Grafische Darstellung der Messdaten von $p$.}
  \label{fig:plotdrehevak}
\end{figure}

\noindent
Durch die Regressionsparameter $m$ kann erneut für jeweils beide Bereiche über $S = -mV$ das Saugvermögen $S$ berechnet werden.

\noindent
Bereich 1: $1.000 \, \si{mbar} \geq p \geq 2 \, \si{mbar}$\\
\begin{align*}
	m_1=& -\SI{0,0296 \pm 0,0004}{1/s}\\
	S_1=& \SI{1,01 \pm 0,11}{l/s}
\end{align*}
Bereich 2: $2 \, \si{mbar} \geq p \geq 1,5 \cdot 10^{-1} \, \si{mbar}$\\
\begin{align*}
	m_2=& -\SI{0,0058 \pm 0,0002}{1/s}\\
	S_2=& \SI{0,197 \pm 0,021}{l/s}
\end{align*}