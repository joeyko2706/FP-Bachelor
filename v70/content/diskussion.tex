\section{Diskussion}
\label{sec:Diskussion}

Zusammenfassend kann gesagt werden, dass die Messungen ohne Schwierigkeiten verliefen und die Messergebnisse den Erwartungen entsprechen. Lediglich bei der
Leckratenmessung der Drehschieberpumpe lieferten die drei identischen Messungen stark variierende Ergebnisse.
Dies ist mit hoher Wahrscheinlichkeit auf eine nicht ausreichende Genauigkeit beim Einstellen des Gleichgewichtsdrucks
zurückzuführen.
\noindent
Die bestimmten Werte für das Saugvermögen der Turbomolekularpumpe sind in \autoref{tab:abw_turbo} zusammen mit der relativen Abweichung zum Wert,
der vom Hersteller angegeben wurde, abgebildet. Für die Turbomolekularpumpe sind dies $S = \SI{77}{l/s}$. Von der Evakuierungsmessung wurde hierbei lediglich der
Wert des ersten Bereiches verwendet.

\begin{table}[H]
    \centering
    \small
    \begin{tabular}{S [table-format=5.0]  c c}
     \toprule
     {Verfahren} & $\text{S} \mathbin{\scalebox{1.5} / } \si{\litre\per\second}$ & $\text{relative Abweichung} \mathbin{\scalebox{1.5} / } \si{\percent}$ \\
     \midrule
     \text{Evakuierung}                            &  10,0 \pm 1,1      & 87,01 \\
     \text{Leck $\SI{5e-5}{\milli\bar}$}         & 21 \pm 7            & 72,73 \\
     \text{Leck $\SI{7e-5}{\milli\bar}$}          & 27 \pm 9             &  64,94 \\
     \text{Leck $\SI{1e-4}{\milli\bar}$}         & 34 \pm 14             & 55,84 \\
     \text{Leck $\SI{2e-4}{\milli\bar}$}         & 38 \pm 13            &  50,65 \\
    \bottomrule
    \end{tabular}
    \caption{Saugvermögen und relative Abweichungen zur Herstellerangabe der Turbomolekularpumpe.}
    \label{tab:abw_turbo}
\end{table} 

\noindent
Es fällt auf, dass die Messwerte allesamt stark vom angegebenen Wert abweichen, dabei aber für einen steigenden Gleichgewichtsdruck zu Beginn der Messung immer
mehr dem Herstellerwert entsprechen. Abweichungen könnten hierbei auf den verwendeten Aufbau zurückzuführen sein, bei dem der Druck weit entfernt von der Pumpe
gemessen wurde. Es muss hierbei der Leitwert des Vakuumgefäßes beachtet werden.
\newline
Die Werte des Saugvermögens der Drehschieberpumpe sind ebenfalls in \autoref{tab:abw_dreh} zusammen mit ihren Abweichungen dargestellt. Das Saugvermögen der Pumpe
ist vom Hersteller als $S = \SI{1,1}{l/s}$ gegeben.

\begin{table}[H]
    \centering
    \small
    \begin{tabular}{S [table-format=5.0]  c c}
     \toprule
     {Messverfahren} & $\text{S} \mathbin{\scalebox{1.5} / } \si{\litre\per\second}$ & $\text{relative Abweichung} \mathbin{\scalebox{1.5} / } \si{\percent}$ \\
     \midrule
     \text{Evakuierung}                 & 1,01 \pm 0,10            &  8,18 \\
     \text{Leck $\SI{0,5}{\milli\bar}$} & 0,82 \pm 0.09           &  25,45 \\
     \text{Leck $\SI{10}{\milli\bar}$}  & 0,99 \pm 0,32            & 10,0 \\
     \text{Leck $\SI{50}{\milli\bar}$}  & 1,2 \pm 0,4            & 9,09 \\
     \text{Leck $\SI{100}{\milli\bar}$}  & 1,1 \pm 0,4            & 0 \\
    \bottomrule
    \end{tabular}
    \caption{Saugvermögen und relative Abweichungen zur Herstellerangabe der Drehschieberpumpe.}
    \label{tab:abw_dreh}
\end{table}

\noindent
Hierbei wurde von der Evakuierungsmessung erneut nur der Wert des ersten Bereichs berücksichtigt.
Die berechneten Werte liegen insgesamt sehr viel näher an den Herstellerangaben, als die der Turbomolekularpumpe und stimmen
im Rahmen des Messfehlers mit den Herstellerangaben überein.
\newline
Die Druckabhängigkeit des Saugvermögens beider Pumpen ist in den folgenden Abbildungen graphisch dargestellt.

\begin{figure}[H]
    \centering
    \includegraphics[width=0.7\textwidth]{build/saug1.pdf}
    \caption{Grafische Darstellung der Druckabhängigkeit des Saugvermögens der Turbomolekularpumpe.}
  \end{figure}

  \begin{figure}[H]
    \centering
    \includegraphics[width=0.7\textwidth]{build/saug2.pdf}
    \caption{Grafische Darstellung der Druckabhängigkeit des Saugvermögens der Drehschieberpumpe.}
  \end{figure}








