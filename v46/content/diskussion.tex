\section{Diskussion}
\label{sec:Diskussion}


Die Messung der magnetischen Flussdichte lief, wie zu erwarten, problemlos ab und das Ergebniss entspricht den Erwartungen.\newline
\noindent
Bei den weiterführenden Messungen gab es jedoch Schwierigkeiten. Bei den Messungen der Rotationswinkel konnten mit dem Interferenzfilter mit $\lambda = 2,51 \si{\micro\metre}$ 
bei beiden Dotierten Proben keine Messwerte aufgenommen werden. Dies lag daran, dass die Signalspannung sich bei variierendem Winkel am Goniometer nicht zu verändern schien.
Dies machte es unmöglich, einen Winkel zu finden, für den sie minimal würde.
\newline
Auch bei den restlichen Messwerten gab es teilweise Probleme beim Finden der minimalen Signalspannung, da sich diese für einen großen Winkelbereich nicht verändert hat und so
kein einzelner Winkel für das Minimum erkennbar war. In diesen Fällen musste ein Winkel aus diesem Bereich gewählt werden, was teilweise die große Streuung der Werte erklären könnte.
\newline
Die final resultierenden Werte für die effektive Masse im Verhältnis zur Elektronmasse lauten
\begin{align*}
    m_1^* = (4,5 \pm 0,7) \cdot 10^{-32} \, \si{\kilo\gram} = (0,050 \pm 0,007) \cdot m_\text{e} \\
    m_2^* = (7,4 \pm 1,0) \cdot 10^{-32} \, \si{\kilo\gram} = (0,081 \pm 0,011) \cdot m_\text{e}.
\end{align*}
Der zu erwartende Wert liegt bei $m = 0,063 \cdot m_\text{e} $. Dies ergibt eine relative Abweichung von 20,6\% und 28,6\%.

