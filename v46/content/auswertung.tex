\section{Auswertung}
\label{sec:Auswertung}


\subsection{Messung der magnetischen Flussdichte}
\label{subsec:Flussdichte}

Die Werte von der Messung der magnetischen Flussdichte sind \autoref{tab:fluss} zu entnehmen.


\begin{table}[H]
  \caption{Messwerte der magnetischen Flussdichte.}
  \label{tab:fluss}
  \centering
  \begin{tabular}{c c}
      \toprule
      Abstand [$\si{\milli\metre}$] & Flussdichte [$\si{\milli\tesla}$]\\
      \midrule
      -10 & 277 \\
      -9 & 311 \\
      -8 & 344 \\
      -7 & 368 \\
      -6 & 384 \\
      -5 & 393 \\
      -4 & 401 \\
      -3 & 406 \\
      -2 & 409 \\
      -1 & 411 \\
      0 & 411 \\
      1 & 410 \\
      2 & 408 \\
      3 & 405 \\
      4 & 399 \\
      5 & 390 \\
      6 & 379 \\
      7 & 364 \\
      8 & 333 \\
      9 & 303 \\
      10 & 253 \\
      \bottomrule
    \end{tabular}
\end{table}

\noindent
Die Messwerte sind in Folgender \autoref{fig:fluss} graphisch dargestellt.

\begin{figure}[H]
  \centering
  \includegraphics{plot-Fluss.pdf}
  \caption{Messwerte der magnetischen Flussdichte.}
  \label{fig:fluss}
\end{figure}

\noindent
Der maximale Wert für die magnetische Flussdichte kann als $\SI{411}{\milli\tesla}$ abgelesen werden.

%%%%%%%%%%%%%%%%%%%%%%%%%%%%%%%%%%%%%%%%%%%%%%%%%%%%%%%%%%%%%%%%%%%%%%%%%%%%%%%%%%%%%%%%%%%%%%%%%%%%%%%%%%%%%%%%%%%%%
\subsection{Bestimmung der Rotationswinkel}
\label{subsec:Rotationswinkel}

Die Eigenschaften der drei verwendeten Proben sind in \autoref{tab:proben} aufgeführt.


\begin{table}[H]
  \caption{Eigenschaften der verwendeten Proben.}
  \label{tab:proben}
  \centering
  \begin{tabular}{l | c c c}
      & 1.Probe & 2.Probe & 3.Probe \\
      \hline
      Dotierung $N$ [$\si{\per\cubic\centi\metre}$] & - & $1,2 \cdot 10^{18}$   & $2,8 \cdot 10^{18}$ \\
      Länge $L$ [$\si{\milli\metre}$] & 5,11 & 1,36 & 1,296 \\
    \end{tabular}
\end{table}

\noindent
Die Messwerte der Drehwinkel sind in den folgenden Tabellen für die jeweiligen Proben dargestellt. $\theta_{1}$ und $\theta_{2}$ beschreiben hierbei die gemessenen Winkel bei
entgegengesetzt gepolten Magnetfeldern. Ebenfalls in der Tabelle aufgeführt sind die zugehörigen Rotationswinkel die durch 
\begin{align*}
  \theta = \frac{1}{2} (\theta_1 - \theta_2)
\end{align*}
\noindent
folgen. Die Rotationswinkel werden hierbei mit Hinblick auf weitere Rechnungen im Bogenmaß angegeben und auf die jeweilige Länge $L$ der zugehörigen Probe normiert.

\begin{table}[H]
  \centering
  \begin{tabular}{c c c c}
    $\lambda/\si{\micro\meter}$ & $\theta_{1}/\si{\degree}$ & $\theta_{2}/\si{\degree}$ & $\theta/d/\symup{rad}\, \si{m}^{-1}$\\
    \midrule
    1,06  &  129,42  &  105,75  &  40,5 \\
    1,29  &  115,33  &  108,83  &  11,12 \\
    1,45  &  127,83  &  109,83  &  30,8 \\
    1,72  &  121,33  &  111,67  &  16,54 \\
    1,96  &  120,83  &  112,75  &  13,83 \\
    2,156 &  124,0   &  116,0   &  13,69 \\
    2,34  &  125,58  &  121,33  &  7,27 \\
    2,51  &  135,17  &  130,0   &  8,84 \\
    2,65  &  121,42  &  119,67  &  2,99 \\
    \bottomrule
  \end{tabular}
  \caption{Messwerte der zwei Winkel für die jeweilige Wellenlänge mit der zugehörigen Winkeldifferenz in Radiant für die undotierte Probe.}
  \label{tab:m1}
\end{table}

\begin{table}[H]
  \centering
  \begin{tabular}{c c c c}
    $\lambda/\si{\micro\meter}$ & $\theta_{1}/\si{\degree}$ & $\theta_{2}/\si{\degree}$ & $\theta/d/\symup{rad}\, \si{m}^{-1}$\\
    \midrule
    1,06  &  121,08  &  112,25  &  56,68 \\
    1,29  &  112,75  &  121,25  &  54,54 \\
    1,45  &  118,75  &  116,0   &  17,65 \\
    1,72  &  119,0   &  116,83  &  13,9 \\
    1,96  &  119,17  &  111,33  &  50,26 \\
    2,156 &  116,92  &  127,08  &  65,24 \\
    2,34  &  131,25  &  129,92  &  8,56 \\
    2,51  &  0,0     &  0,0     &  0,0 \\
    2,65  &  104,75  &  123,33  &  119,24 \\
    \bottomrule
  \end{tabular}
  \caption{Messwerte der zwei Winkel für die jeweilige Wellenlänge mit der zugehörigen Winkeldifferenz in Radiant für die $1,2 \cdot 10^{18} \si{\per\cubic\centi\metre}$ Probe.}
  \label{tab:m1}
\end{table}

\begin{table}[H]
  \centering
  \begin{tabular}{c c c c}
    $\lambda/\si{\micro\meter}$ & $\theta_{1}/\si{\degree}$ & $\theta_{2}/\si{\degree}$ & $\theta/d/\symup{rad}\, \si{m}^{-1}$\\
    \midrule
    1,06  &  122,5   &  108,92  &  91,46 \\
    1,29  &  118,67  &  110,0   &  58,36 \\
    1,45  &  120,08  &  112,92  &  48,26 \\
    1,72  &  119,17  &  108,83  &  69,58 \\
    1,96  &  121,75  &  109,33  &  83,61 \\
    2,156 &  122,5   &  111,83  &  71,82 \\
    2,34  &  122,67  &  121,25  &  9,54 \\
    2,51  &  0,0     &  0,0     &  0,0 \\
    2,65  &  127,33  &  108,33  &  127,94 \\
    \bottomrule
  \end{tabular}
  \caption{Messwerte der zwei Winkel für die jeweilige Wellenlänge mit der zugehörigen Winkeldifferenz in Radiant für die $2,8 \cdot 10^{18} \si{\per\cubic\centi\metre}$ Probe.}
  \label{tab:m1}
\end{table}
\noindent
Die sich daraus ergebenden Winkeldifferenzen sind in \autoref{fig:diffwerte} gegen $\lambda^2$ graphisch dargestellt.

\begin{figure}[H]
  \centering
  \includegraphics{plot-Werte.pdf}
  \caption{Messwerte der Rotationswinkel gegen jeweilige Wellenlänge $\lambda^2$ für alle drei Proben.}
  \label{fig:diffwerte}
\end{figure}


\subsection{Bestimmung der effektiven Masse}
\label{subsec:masse}

Um die effektive Masse zu bestimmen, werden zunächst von den Werten der dotierten Proben, die der hochreinen GaAs Probe subtrahiert, um die Effekte der Leitungselektronen
auf den Rotationswinkel zu isolieren. Die Differenzen werden anschließend, wie zuvor, gegen $\lambda^2$ aufgetragen und eine lineare Ausgleichsrechnung durchgeführt.

\begin{figure}[H]
  \centering
  \includegraphics{plot-linReg.pdf}
  \caption{Differenzwerte der Rotationswinkel von Dotierten und Undotierten Proben mit Ausgleichsgeraden.}
  \label{fig:diffwerte2}
\end{figure}

\noindent
Die linearen Ausgleichsrechnungen der Form

\begin{align}
  \label{eq:mulm}
  \theta_\text{frei} (\lambda^2) = a \cdot \lambda^2
\end{align}

\noindent
liefern die Werte

\begin{align*}
  a_1 = (15,6 \pm 4,56) \, \si{\meter}^{-3} \\
  a_2 = (13,6 \pm 3,83) \, \si{\meter}^{-3},
\end{align*}
\noindent
wobei zu beachten ist, dass die Werte für $\lambda = 2,34 \, \si{\micro\metre}$ und $\lambda = 2,51 \, \si{\micro\metre}$ in der Rechnung nicht berücksichtigt wurden.
Dies liegt daran, dass sie sehr weit vom Trend der anderen Werte abweichen und somit als Ausreißer zu betrachten sind.
Es kann nun durch einsetzen von \autoref{eq:mulm} in \autoref{eq:effmass} die Gleichung
\begin{equation*}
  m^{*} = \sqrt{\frac{e_0^3}{8\pi^2 \varepsilon_0 c^3} \frac{N B}{n} \frac{1}{a}}
\end{equation*}
\noindent
zur Bestimmung der effektiven Masse gefunden werden. Für die Berechnung werden folgende Werte verwendet.
\begin{align*}
  N_1 &= 1,2 \cdot 10^{12} \, \si{\meter}^{-3} \\
  N_2 &= 2,8 \cdot 10^{12} \,\si{\meter}^{-3} \\
  a_1 &= (15,6 \pm 4,56) \, \si{\meter}^{-3} \\
  a_2 &= (13,6 \pm 3,83) \, \si{\meter}^{-3} \\
  B &= 0,411 \, \si{\tesla} \\
  n &= 3,354 \\
  e_0 &= 1,602 \cdot 10^{-19} \, \si{\coulomb} \\
  \varepsilon_0 &= 8,854 \cdot 10^{-12} \, \si{\ampere\second\per\volt\per\metre} \\
  c &= 2,998 \cdot 10^{8} \, \si{\metre\per\second} \\
\end{align*}
\noindent
Die Werte liefern für die effektive Masse die Ergebnise

\begin{align*}
  m_1^* = (4,5 \pm 0,7) \cdot 10^{-32} \si{\kilo\gram} \\
  m_2^* = (7,4 \pm 1,0) \cdot 10^{-32} \si{\kilo\gram}.
\end{align*}

