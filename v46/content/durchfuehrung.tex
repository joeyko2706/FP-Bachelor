\section{Durchführung}
\label{sec:Durchführung}

\section{Durchführung}
\label{sec:Durchführung}

\subsection{Versuchsaufbau}
\label{subsec:Versuchsaufbau}

Der Versuch wird nach \autoref{fig:Aufbau} aufgebaut.
\begin{figure}[H]
	\centering
    \includegraphics[width=0.6\linewidth]{data/PolarisationDrehung.png}
	\caption{Skizze des Versuchsaufbaus .\cite{Anleitung46}}
	\label{fig:Aufbau}
\end{figure}

Als Lichtquelle wird eine Halogenlampe verwendet, dessen Lichtspektrum zum 
Teil im nahen Infrarotbereich liegt. 
Das emittierte Licht wird mithilfe einer Kondensatorlinse gebündelt. 
Danach wird der Lichtstrahl mithilfe eines Lichtzerhackers in Bündel 
geteilt,
was das Rauschen der Photowiderstände aufgrund ihrer hohen 
Innenwiderstände reduziert. Der Lichtzerhacker ist an einen 
Selektivverstärker angeschlossen und verwendet dieselbe Mittenfrequenz,
wie es der Lichtzerhacker tut. Hinter dem Lichtzerhacker befindet sich ein 
Glan-Thomson-Prisma aus Kalkspat.
Das Licht trifft danach auf eine scheibenfömige Probe, die sich innerhalb eines Magnetfeldes befindet, dessen
Magnetfeldlinien parallel zur Ausbreitungsrichtung des Lichtes sind. Hinter dem Elektromagneten können verschiedene Interferenzfilter
geschaltet werden. Um die Rotation der Polarisationsebene messen zu können, wird der Lichtstrahl mithilfe eines zweiten Glan-Thomson-Prismas
in zwei Teilstrahlen aufgeteilt, dessen Polarisation orthogonal zueinander steht. Das Licht wird erneut durch Linsen gebündelt und die Intensität mithilfe von
Photowiderständen gemessen. Die Signale der Photowiderstände werden an einen Differenzverstärker angeschlossen, dessen Ausgang
an einem Oszilloskop angezeigt wird.

\subsection{Justierung der Apparatur}
\label{subsec:Justierung}

Um die Apparatur zu Justieren wird zunächst die Probe und der Interferenzfilter aus der Vorrichtung herausgenommen. Nun wird der Strahlengang so eingestellt, dass das
meiste Licht bis zum Ende des Strahlenganges durchkommt und somit die Hauptintensität innerhalb der Vorrichtung liegt.
Als nächstes muss überprüft werden, ob die Polarisationsvorrichtung funktioniert. Dazu wird das Prisma so eingestellt werden, dass die Lichtintensität von einem 
der beiden Strahlgänge vollständig verschwindet.
Nun wird der Lichtzerhacker auf \SI{450}{\hertz} eingestellt und die Mittenfrequenz beim Selektivverstärker auf denselben Wert gestellt.
Der Photowiderstand, auf den auch das Licht gestrahlt wird, wird mit dem Selektivverstärker über den Kanal "Input" verbunden. Der Ausgang "Resonance"
wird an das Oszilloskop angeschlossen. Mithilfe der Frequenzstellknöpfe am Selektivverstärker wird das Signal so gewählt, das die Amplitude maximal wird.
Es ist der Gütefaktor auf den Maximalwert $Q=100$ einzustellen.

\subsection{Messprogramm}
\label{subsec:Messprogramm}

Es wird nun mit einer reinen Probe GaAs die Faraday-Rotation gemessen, indem das Glan-Thomson-Prisma bei maximaler Magnetfeldstärke so eingestellt wird,
das das Signal auf dem Oszilloskop minimal wird. Nachdem der Winkel des Prismas gemessen wurde, kann das Magnetfeld langsam heruntergedreht, umgepolt und wieder auf
den Maximalwert hochgedreht werden. Die Messung wird mit allen Interferenzfiltern wiederholt und dient später als Referenz.
Der Vorgang wird mit zwei Proben dotierten GaAs mit unterschiedlichen Dichten dotieren Materials wiederholt.
Zum Schluss wird die magnetische Flussdichte in Richtung des einfallenden Lichtes mit Hilfe einer Hall-Sonde bei maximalem Magnetfeldstrom vermessen.